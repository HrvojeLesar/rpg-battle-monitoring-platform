\documentclass{foi}
\usepackage[utf8]{inputenc}
\usepackage{lipsum}

\vrstaRada{\diplomski} % \diplomski
\title{Izrada platforme za praćenje bitke u igri igranja uloga}

\author{Hrvoje Lesar}
\spolStudenta{\musko} % \zensko ili \musko
\mentor{Markus Schatten}
\spolMentora{\musko} % \zensko ili \musko
\godina{2025}
\mjesec{rujan}
\date{2025}
%\status{redoviti}
\indeks{0016133479}
\smjer{Organizacija poslovnih sustava} % (ili Poslovni sustavi, Ekonomika poduzetništva, Primjena informacijske tehnologije u poslovanju, Informacijsko i programsko inženjerstvo, Baze podataka i baze znanja, Organizacija poslovnih sustava, Informatika u obrazovanju)
\titulaProfesora{Prof. dr. sc.}

\sazetak{Opsega od 100 do 300 riječi. Sažetak upućuje na temu rada, ukratko se iznosi čime se rad bavi, teorijsko-metodološka polazišta, glavne teze i smjer rada te zaključci.}

\kljucneRijeci{riječ; riječ; ...riječ; Obuhvaća 7+/-2 ključna pojma koji su glavni predmet rasprave u radu.}

\begin{document}

\maketitle

\tableofcontents

\pagestyle{plain}

\chapter{Uvod}
\label{chapter:uvod}

\begin{itemize}
    \item Rpg
    \item Cilj
    \item Prvi dio...
    \item Praktični dio
\end{itemize}

\chapter{Igre igranja uloga}
\label{chapter:igra-igranja-uloga}

Igre igranja uloga predstavljaju jedan od najsloženijih i kreativnih spojeva
priče, izvedbe i igre u suvremenoj kulturi. 
Ovakva igra poziva igrače da zajednički stvaraju i nastanjuju zamišljeni svijet,        
vođeni kombinacijom strukturiranih pravila i spontanog pripovijedanja.
Svaka sesija je istovremeno i igra i priča, a njezin ishod oblikuju
mašta igrača i slučajnost ishoda bacanja kockica.

Igre igranja uloga nemaju jednu definiciju i možemo ih definirati s različitih gledišta:
\begin{itemize}
    \item Situacija igranja uloga definirana kao situacija u kojoj se od pojedinca izričito traži da preuzme
        ulogu koja inače nije njegova, ili ako jest njegova, onda u okruženju koje nije
        uobičajeno za izvođenje te uloge. \cite{mann1956experimental}
    \item Igranje uloga nije jedinstvena jasno definirana aktivnost, već čitav niz aktivnosti
        okupljenih pod pogodnim nazivom. Na jednom kraju spektra nalazi se intenzivno odražavanje
        sobnih emocija, dok se na drugom kraju nalazi situacija u kojoj je preuzimanje uloge
        bliže konceptu zagovaranja. \cite{white2024tabletop}
    \item Igranje uloga je umjetnost iskustva, a stvaranje igra uloga znači kreiranje novih iskustava. \cite{white2024tabletop}
    \item Igranje uloga definira se kao bilo koji čin u kojem se istovremeno stvara, dodaje i promatra imaginarna stvarnost. \cite{white2024tabletop}
    \item Igra igranja uloga mora se sastojati od interaktivnog pripovijedanja: 
        sposobnosti likova i razrješenja radnji, definirani su brojevima ili količinama, kojima se
        manipulira prema određenim pravilima. Donošenje odluka igrača pokreće i pomiče priču u naprijed.
        Uz skupinu koja djeluje kao autor, priča organski raste i odigrava se, bivajući doživljena
        od svojih stvaratelja. \cite{white2024tabletop}
\end{itemize}

\section{Ratne igre}
\label{section:ratne-igre}

Početak i inspiracija za igre igranja uloga dolazi iz ratnih igara.
Nasljeđe ratnih igara seže od šaha, budući da su prve igre unutar posebne
kategorije ratnih igara uvelike posuđivale ploče, figure i mehanike upravo
iz šaha. Georg von Rei{\ss}witz se smarta ocem ratnih igara, jer je razvio
prvi sustav ratnih igara koji je široko korišten ako ozbiljan alat za obuku
i istraživanje. Razvijenu igru su nazvali tzv. \texttt{Das Kriegsspiel} te je igra
zadovoljila dugo prepoznatu potrebu za jeftinim i lako ponovljivim sredstvom
za obuku časnika u zapovijedanju i planiranju bitke. Vojne ratne igre gotovo se uvijek
fokusiranju na sadašnjost, na ondašnje vojske, tehnologiju i države u vrijeme
njihova nastanka. Veliki dio razvoja ratnih igara osamnaestog
i devetnaestog stoljeća održava neprestana poboljšanja u sredstvima i 
provedbi ratovanja te posljedičnu potrebu da se ratne igre stalno usklađuju s
realnošću. \cite{peterson2012playing}

Tek kad su se hobisti počeli poigravati ovim sustavima, uspjeli su osloboditi
ratne igre ograničenja suvremenog konteksta i istražiti povijesna razdoblja,
buduće moguće svjetove, pa čak i nemoguće fantastične svjetove.
Hobisti su također odbacili strogo reproduciranje stvarnih uvjeta na
bojištu u korist više uravnoteženijeg pristupa koji je kombinirao
realističnost i igrivost. Do 1960-ih, ovi zaigrani ljubitelji ratnih igara
transformirali su ih iz sredstva za vojnu obuku u znatno maštovitiju aktivnost,
onu koja je mogla poslužiti kao temelj za modeliranje događaja u igri
poput Dungeons \& Dragons (skraćeno D\&D). \cite{peterson2012playing}

\section{Nastanak sustava Dungeons \& Dragons}
\label{section:nastanak-sustava-dungeons-dragons}

Dungeons \& Dragons (kraće poznato kao D\&D), igra nastala 1974.
godine predstavlja ključni trenutak u povijesti igra uloga.
Igru su razvili Gary Gygax i Dave Arneson, koji su težili proširenju
mogućnosti stolnih ratnih igara prema novim narativnim i
imaginativnim domenama. Prije D\&D-a, Gygax je već imao utjecaj
u wargaming zajednici, osobito kroz svoj rad na igri \texttt{Chainmail},
srednjovjekovnoj miniaturskoj ratnoj igri koju je razvio i bio jedan
od autora 1971. godine. 
\texttt{Chainmail} je izvorno zamišljen kao skup pravila za simulaciju
srednjovjekovnih bitaka s minijaturama, no uključivao je i fantastični dodatak
koji je omogućava igračima da u igru unesu mistična bića i čarobne elemente.
Ova dopuna održavala je Gygaxovu fascinaciju srednjovjekovnom literaturom i
fantastičnim narativima, posebno djelima J.R.R Tolkeina i 
Roberta E. Howarda. \cite{sidhu2024fifty}

Dave Arneson je eksperimentirao s narativnim pristupom u svojoj kampanji
Blackmoor, u kojoj su se pojedinačni likovi mogli razvijati kroz
više sesija i aktivno utjecati na tijek priče, za razliku od
tradicionalnog wargaminga u koje igrači kontroliraju cijele jedinice ili vojske.
Arnesonove inovacije, u kombinaciji s Gygaxovim strukturiranim sustavom pravila
iz Chainmaila, stvorile su temelje za mehaniku i narativni potencijal igre
D\&D. \cite{sidhu2024fifty}

Prvo izdanje D\&D-a je izdano 1974. godine, bilo je u početku skromno 
po opsegu, no uvelo je revolucionarne koncepte;
igrači su mogli preuzeti uloge različitih likova istraživati fantastične tamnice,
sudjelovati u zadacima vođenim od strane Dungeon Mastera (DM-a),
koji moderira svijet igre.
Pravila D\&D-a uključivala su elemente miniaturskog wargaminga,
vjerojatnosnih mehanika s kockama i suradničkog pripovijedanja,
stvarajući novi oblik interaktivne zabave koji je spajao strategiju,
maštu i narativ. Tijekom vremena igra se razvila kroz više izdanja,
od kojih je svako unaprijedilo pravila i proširilo mogućnosti, no osnovni
princip suradničkog pripovijedanja i igre vođene likovima je ostao nepromijenjen.

\chapter{Opis pravila igre - Dungeons \& Dragons}
\label{chapter:opis-pravila-igre-dungeons-dragons}

U ovome poglavlju će biti opisana pravila igre D\&D-a.
Pravila igre su se razvila od početne koncepcije igre 70-tih godina 20. stoljeća.
Poglavlje će biti usredotočeno na pravila definirana u \texttt{Player's Handbook 2014}.
Svakim novim izdanjem igre pravila se suptilno nagorađuju, poboljšavaju i razvijaju, te se 
dodaju nove mehanike igre. Ovdje definiramo zadnje izdanje koje će se poglavlje i kasnija
implementacije koristiti i pratiti.

Dungeons \& Dragons je igra igranja uloga posvećena pripovijedanju u svjetovima
mača i čarolije. Dijeli elemente s dječjim igrama pretvaranja. Kao i te igre, D\&D pokreće
mašta. Radi se o zamišljanju visokog dvorca pod olujnim noćnim nebom i predočavanju kako bi
pustolov mogao reagirati na izazove koje taj prizor predstavlja.
Za razliku od igra pretvaranja, D\&D pričama daje strukturu, način određivanja posljedica
za akcije igrača. Igrači bacaju kockice kako bi razriješili jesu li njihovi napadi pogodili
ili promašili, ili mogu li se njihovi pustolovi popeti na liticu, izmaknuti se udaru čarobne munje
ili izvesti neki drugi opasan pothvat. Sve je moguće, ali kockice čine neke ishode vjerojatnijima
od drugih. \cite{playershandbook2014}

Jedan igrač preuzima ulogu Dungeon Mastera, glavnog pripovjedača i suca igre.
DM stvara pustolovine za likove, koji se kreću kroz njihove opasnosti i odlučuje koje
će putove istražiti. Više o DM-u će biti u potpoglavlju \ref{section:dungeon-master}.

Svaki igrač stvara pustolova (koji se naziva i likom) i udružuje se s drugim
pustolovima. Radeći zajedno, grupa može istraživati mračnu tamnicu, ukleti dvorac,
izgubljeni hram... Pustolovi mogu rješavati zagonetke, razgovarati s drugim
likovima, boriti se protiv čudovišta i otkrivati čudesne čarobne predmete i blago.

\section{Lista pojmova}
\label{section:lista-pojmova}

Ovo poglavlje će služiti za definiranje terminologije i kratica koje će se
koristiti kroz rad. Svaki termin će biti kraće opisan i potencijalno imati
engleski naziv (koji je uglavnom više prepoznatljiv nego hrvatski prijevod),
no cijeli kontekst ne mora biti razumljiv u spisu, već će biti opisan u nekom od sljedećih poglavlja.

\begin{itemize}
    \item \textbf{D\&D} Dungeons \& Dragons
    \item \textbf{DM} Dungeon Master
    \item \textbf{NPC} Non-Player Character
    \item \textbf{D20} Kockica s dvadeset strana
    \item \textbf{D2, D4, D6, D8, D10, D12} Kockice s određenim brojem strana
    \item \textbf{AC} Armour class, klasa oklopa
    \item \textbf{DC} Difficulty class, težina zadatka
    \item \textbf{HP} Hit points, životni bodovi
    \item \textbf{XP} Experience points
    \item \textbf{Ability score} Sposobnosti
    \item \textbf{Hit die} Kockica za životne bodove
\end{itemize}

\section{Dungeon Master}
\label{section:dungeon-master}

Dungeon master je kreativna sila iza D\&D sesije. DM stvara svijet koji ostali igrači
istražuju, te također osmišljava i vodi pustolovine koje pokreću priču.
Pustolovina se obično temelji na uspješnom dovršetku neke potrage i može trajati
kratko poput jedne sesije ili se protezati kroz više sesija.
Duže pustolovine mogu uvući igrače u velike sukobe za čije je rješavanje potrebno
više sesija. Kada se povežu u niz pustolovine čine trajnu priču nazvanu kampanja.
D\&D kampanja može uključivati desetke pustolovina i trajati jednu sesiju, mjesecima ili
pa ćak i godinama. \cite{dungeonmastersguide2014}

Dungeon master preuzima mnogo različitih uloga. Kao arhitekt kampanje, DM stvara pustolovine
postavljajući pozicije čudovišta, zamka i blaga koje likovi drugih igrača mogu otkriti.
Kao pripovjedač, DM pomaže ostalim igračima vizualizirati što se događa oko njih,
improvizirajući kada pustolovi učine nešto ili odu negdje neočekivano. Kao glumac, DM
igra uloge čudovišta i sporednih likova, time im daje život.
A kao sudac, DM tumači pravila i odlučuje kada ih se treba pridržavati, a kada ih promijeniti.
Svaki DM pristupa ulogama izmišljanja, pisanju, pripovijedanju, improvizaciji, glumi i suđenju
drugačije i vjerojatno će igrači i sam DM u nekim uživati više nego u drugima.

D\&D pravila pomažu DM-u i ostalim igračima da se dobro zabave, ali pravila nisu glavni autoritet.
DM je glavni i upravlja igrom te smije "kršiti" ili mijenjati definirana pravila. 
Ipak, cilj DM-a nije pobiti pustolove, već stvoriti svijet kampanje koji se vrti oko njihovih
djela i odluka, te osigurati da se igrači vraćaju po još.

Uspjev D\&D sesije ovisi o sposobnosti DM-a da zabavi ostale igrače za stolom.
Dok je njihova uloga stvaranje likova, udahnuti im život i pomoći usmjeravati kampanju
kroz postupke svojih likova. DM treba održati igrače (i sebe) zainteresiranima i uživljenima
u svijet koji je stvorio te omogućiti njihovim likovima da čine nevjerojatne stvari.
Poznavanje u kojim dijelovima D\&D-a igrači najviše uživaju pomaže DM-u stvoriti i voditi
avanture u kojima će uživati i kojih će se sjećati. \cite{dungeonmastersguide2014}

Aktivnosti na koje se DM može fokusirati za bolje zadovoljstvo igrača \cite{dungeonmastersguide2014}:
\begin{itemize}
    \item Gluma; igrači koji uživaju u glumi vole ulaziti u ulogu svog lika i govoriti njihovim klasom.
        Kao roleplayeri u duši, oni uživaju u društvenim interakcijama s NPC-ovima, čudovištima i ostalim
        članovima družine.
    \item Istraživanje; igrači koji žude za istraživanjem žele iskusiti čuda koja nudi fantastični svijet.
        Žele znati što se nalazi iza sljedećeg ugla ili brda, također vole pronalaziti skrivene tragove i blago
    \item Poticanje akcije; neki igrači vole poticati akciju, željni su da se stvari događaju čak
        i ako to znači preuzimanje opasnih rizika. Rađe će srljati u opasnost i suočiti se s posljedicama
        nego se suočiti s dosadom.
    \item Borba; drugi igrači uživaju u fantastičnoj borbi poput premlaćivanja zlikovaca i čudovišta.
        Traže bilo kakav izgovor da započne borba, preferirajući akciju nad pažljivim promišljanjem.
    \item Optimizacija lika; igrači koji uživaju u optimizaciji sposobnosti svojih likova vole
        podešavati svoje likove za vrhunske borbene performanse stjecanjem razina, novih značajki i
        čarobnih predmeta, rado će prihvatiti svaku priliku da pokažu nadmoć svojih likova
    \item Rješavanje problema; igrači koji žele rješavati probleme vole proučavati motivacije
        drugih likova, razmrsiti spletke zlikovaca, rješavati zagonetke i smišljati planove
    \item Pripovijedanje; igrači koji vole pripovijedanje žele doprinijeti priči. Sviđa im se kada su njihovi
        likovi uključeni u priču koja se razvija i uživaju u susretima koji su vezani s
        glavnom radnjom i proširuju je.
\end{itemize}

\section{Sustav D20}
\label{section:sustav-d20}

Dvadeseto strana kockica je najčešće korištena za određivanje uspjeha ili neuspjeha
radnji u D\&D-u i tu kockicu nazivom d20. Ponekad se koriste dvanaesto strane, osmo
strane, šesto strane, četvero strane kockice i novčić, one također imaju nazive
ovisno o broju strana, d12, d10, d8, d6, d4 i d2.

Svi likovi i čudovišta u igri definirani su kroz šest osnovnih sposobnosti: 
snaga (engl. \textit{strength}),
spretnost (engl. \textit{dexterity}),
izdržljivost (engl. \textit{constitution}),
inteligencija (engl. \textit{intelligence}),
mudrost (engl. \textit{wisdom})
i karizma (engl. \textit{charisma}).
Kod većine pustolova vrijednosti tih sposobnosti kreću se od 3 do 18,
dok kod čudovišta mogu biti niske poput 1 ili visoke do 30. 
Te vrijednosti, kao i modifikatori koji iz njih proizlaze, temelj su gotovo
svakoh bacanja d20 kockice

Tri glavne vrste bacanja čine srž pravila igre: provjera sposobnosti,
bacanje za napad i bacanje za spas (engl. \textit{saving throw}). Sve tri radnje slijede nekoliko koraka:
\begin{enumerate}
    \item bacanje kockice i dodavanje modifikatora. Na rezultat bačene d20 kockice dodajemo odgovarajući modifikator.
    Najčešće je to modifikator jedne od šest sposobnosti, a ponekad uključuje i bonus stručnosti koji
    odražava specifičnu vještinu lika.
    \item uračunavanje dodatnih bonusa i kazni. Značajke klase, čarolije, specifične okolnosti ili neki drugi
    efekti mogu dodati bonus na rezultat ili ga umanjiti.
    \item usporedba rezultata s ciljanim brojem. Ako je ukupan zbroj jednak ili veći od ciljnog broj, radnja
    (provjera, napad, spašavanje) je uspješno, u suprotnom, nije uspjela. DM obično određuje ciljne brojeve
    i govori igračima jesi li uspjeli odraditi radnju. DM ne treba otkriti igračima ciljani broj za uspjeh.
\end{enumerate}

Ciljni broj za provjere sposobnosti i spas naziva se težina zadatka ili difficulty class (DC), dok se ciljni
broj za napad naziva klasa oklopa ili Armour Class (AC). Ovo jednostavno pravilo temelj je 
rješavanja većine situacija u D\&D-u.

Kod nekih provjera sposobnosti, bacanja za napad ili bacanja za spas bacanje kockice se modificira
posebnom situacijom koju nazivamo prednost (engl. \textit{advantage}) ili odsutnost prednosti (engl. \textit{disadvantage}).
U ovoj situaciji kod bacanja D20 kockice, baca se dodatna D20 i ovisno o situaciji, ako je prednost ili odsutnost prednosti
odabire se kockica s najvećom vrijednošću ili kockica s najmanjom vrijednošću.

\section{Kreacija lika}
\label{section:kreacija-lika}

U ovom poglavlju će biti prikazana kreacija lika, odabir rase i
klasa i ukratko prikazati pojedinosti koje likovi dobivaju odabirom rase i klase,
te postupak kojim se dobivaju početne sposobnosti.

Na slici ispod se nalazi primjer prazne stranice lista o liku (engl. \textit{character sheet}).
Character sheet koristi igračima za zapisivanje i praćenje karakteristika
njihovih likova. Kako ćemo prolaziti kroz različite dijelove kreacije lika
odgovarajući podaci na listu će biti popunjeni, da bi kroz primjer prošli
kroz kreaciju lika.

Character sheet se sastoji od tri stranice, prva stranica (Slika \ref{fig:character-sheet-p1})
sadrži podatke o sposobnostima lika, njegovu klasu, rasu, vještine, trenutnu opremu, životne
bodove.
Druga stranica (Slika \ref{fig:character-sheet-p2})
ima podatke o izgledu lika, starosti, visini, pozadinskoj priči, blagu.
Omogućuje igraču da detaljno opiše svog lika, da se može lakše preuzeti ulogu lika
te gledati fantastični svijet kroz oči lika.
Dok treća stranica (Slika \ref{fig:character-sheet-p2})
sadrži podatke o čarolijama pripremljenim i znanim čarolijama, te razinama (engl. \textit{levels})
i broj čarolija koje like može baciti prije potrebnog odmora.

\begin{figure}[H]
    \centering
    \includegraphics[width=1\textwidth]{slike/character_sheet_p1.jpg}
    \caption{Character sheet - Prva stranica \cite{charactersheet2014}}
    \label{fig:character-sheet-p1}
\end{figure}

\begin{figure}[H]
    \centering
    \includegraphics[width=1\textwidth]{slike/character_sheet_p2.jpg}
    \caption{Character sheet - Druga stranica \cite{charactersheet2014}}
    \label{fig:character-sheet-p2}
\end{figure}

\begin{figure}[H]
    \centering
    \includegraphics[width=1\textwidth]{slike/character_sheet_p3.jpg}
    \caption{Character sheet - Treća stranica \cite{charactersheet2014}}
    \label{fig:character-sheet-p3}
\end{figure}

% \begin{figure}[!htb]
%     \centering
%     \includegraphics[width=0.9\textwidth]{slike/character_sheet_p1.jpg}
%     \caption{Character sheet \cite{charactershet2014}}
%     \label{fig:character-sheet}
% \end{figure}

\subsection{Rase}
\label{section:rase}

Svaki lik pripada rasi, jednoj od mnogih rasa dostupnih u D\&D-u.
Najčešće rase likova su patuljci, vilenjaci i ljudi.
Neke rase također imaju podrase, kao što su planinski patuljci ili
šumski vilenjaci.
Odabrana rasa doprinosi identitetu lika, uspostavlja opći izgled,
pruža prirodne talente stečene kulturom i precima.
Rasa daje određene rasne osobine, kao što su posebna osjetila, vještine
s određenim oružjem ili alatima, stručnost u jednoj ili više vještina ili 
sposobnost korištenja manjih čarolija. Ove osobine se ponekad podudaraju
i nadopunjuju sposobnosti određenih klasa.
Najčešće rasne osobine su uvećanje jedne ili više sposobnosti,
dob lika, moralni kompas, veličina, brzina, jezici koje lik razumije
i kojima se može služiti i razgovarati, te podrase; Članovi podrase imaju
osobine matične rase uz dodatne osobine definirane podrasom.

Ovdje će biti definirane različite rase, njihove podrase, dob,
veličina, brzina i ostale sposobnosti koje se dobivaju odabirom pojedine rase. 
Kao primjer će za patuljke biti navedeni detaljni podaci o rasi, dok će druge rase
imati skraćenu podatke, razumijevanje svih različitih rasnih osobina nije nužno
za implementaciju ni za razumijevanje igre:
\begin{itemize}
    \item Patuljak (engl. \textit{dwarf}) \cite{playershandbook2014}
        \begin{itemize}
            \item Povećanje sposobnosti: izdržljivost se povećava za 2.
            \item Dob: patuljci sazrijevaju istom brzinom kao i ljudi, ali se smatraju mladima do
                50. godine. Žive oko 350 godina.
            \item Veličina: Srednja, patuljci su visoki između 4 i 5 stopa.
            \item Brzina: brzina hoda je 25 stopa i ne smanjuje se nošenjem teškog oklopa
            \item Dodatna svojstva rase:
                \begin{itemize}
                    \item Vid u mraku (engl. \textit{darkvision}): patuljci vide u mraku i svijetlostno prigušenim uvjetima
                        do 60 stopa.
                    \item Patuljačka otpornost (engl. \textit{dwarven resilience}): prednost pri bacanju spasa protiv
                        otrova i otpornost na otrov.
                    \item Patuljačka borbena obuka (engl. \textit{dwarven combat training}): stručnost u korištenju bojne sjekire (engl. \textit{battleaxe}),
                        sjekiricom (engl. \textit{handaxe}), lakim čekićem (engl. \textit{throwing hammer}) i
                        ratnim čekićem (engl. \textit{warhammer}).
                    \item Stručnost s alatom (engl. \textit{tool proficiency}): stručnost u korištenju jednog od
                        vrsta alata: kovački alat, pivarski pribor, zidarski alat
                    \item Poznavanje kamena (engl. \textit{stonecunning}): na sve provjere inteligencije 
                        (vezane uz vještinu povijest), dodaje se dvostruki bonus stručnosti na provjeru.
                \end{itemize}
            \item Jezici: patuljci govore, pišu i čitaju zajednički (engl. \textit{common}) i patuljački
            \item Podrase patuljaka: brdski patuljak; povećanje mudrosti za 1 i maksimalni
                broj životnih bodova povećava se za 1 po razini. Planinski patuljak; povećanje snage za 2 i
                stručnost s lakim i srednjim oklopom.
        \end{itemize}
    \item Vilenjak (engl. \textit{elf})
        \begin{itemize}
            \item Povećanje sposobnosti: spretnost se povećava za 2.
            \item Dob: vilenjaci sazrijevaju istom brzinom kao i ljudi, ali se smatraju odraslima
                tek oko 100. godine. Žive do 750 godina.
            \item Veličina: Srednja, vilenjaci su visoki od ispod 5 do preko 6 stopa.
            \item Brzina: brzina hoda je 30 stopa.
            \item Tri podrase: visoki vilenjak (engl. \textit{high elf}); 
                Šumski vilenjak (engl. \textit{wood elf}); 
                Tamni vilenjak (engl. \textit{drow}); 
        \end{itemize}

    \item Halfling
        \begin{itemize}
            \item Povećanje sposobnosti: spretnost se povećava za 2.
            \item Dob: halflinzi dosežu odraslu dob s 20 godina i žive do oko 150 godina.
            \item Veličina: Mala, halflinzi su visoki oko 3 stope.
            \item Brzina: brzina hoda je 25 stopa.
            \item Jezici: govore, pišu i čitaju zajednički i halflingški (engl. \textit{halfling}).
        \end{itemize}

    \item Čovjek (engl. \textit{human})
        \begin{itemize}
            \item Povećanje sposobnosti: sve sposobnosti se povećavaju za 1.
            \item Dob: ljudi dosežu odraslu dob u kasnim tinejdžerskim godinama i žive manje od stoljeća.
            \item Veličina: Srednja, ljudi znatno variraju u visini i građi.
            \item Brzina: brzina hoda je 30 stopa.
            \item Jezici: ljudi govore, pišu i čitaju zajednički i jedan dodatni jezik po izboru.
        \end{itemize}

    \item Dragonborn
        \begin{itemize}
            \item Povećanje sposobnosti: snaga se povećava za 2, a karizma za 1.
            \item Dob: brzo rastu, odrasli su s 15 godina i žive oko 80 godina.
            \item Veličina: Srednja, viši su i teži od ljudi, često preko 6 stopa.
            \item Brzina: brzina hoda je 30 stopa.
            \item Jezici: govore, pišu i čitaju zajednički i zmajski (engl. \textit{draconic}).
        \end{itemize}

    \item Gnom (engl. \textit{gnome})
        \begin{itemize}
            \item Povećanje sposobnosti: inteligencija se povećava za 2.
            \item Dob: gnomovi odrastaju oko 40. godine i žive 350 do 500 godina.
            \item Veličina: Mala, visoki su između 3 i 4 stope.
            \item Brzina: brzina hoda je 25 stopa.
            \item Jezici: gnomovi govore, pišu i čitaju zajednički  i gnomski (engl. \textit{gnomish}).
            \item Podvrste: šumski gnom (engl. \textit{forest gnome}) Kameni gnom (engl. \textit{rock gnome}).
        \end{itemize}

    \item Poluvilenjak (engl. \textit{half-elf})
        \begin{itemize}
            \item Povećanje sposobnosti: karizma se povećava za 2 i dvije druge sposobnosti po izboru za 1.
            \item Dob: sazrijevaju kao ljudi, odrasli s 20 godina, žive preko 180 godina.
            \item Veličina: Srednja, otprilike iste veličine kao ljudi.
            \item Brzina: brzina hoda je 30 stopa.
            \item Jezici: poluvilenjaci govore, pišu i čitaju zajednički, vilinski (engl. \textit{elvish}) 
                i jedan dodatni jezik po izboru.
        \end{itemize}

    \item Poluork (engl. \textit{half-orc})
        \begin{itemize}
            \item Povećanje sposobnosti: snaga se povećava za 2, a izdržljivost za 1.
            \item Dob: sazrijevaju brže od ljudi s oko 14 godina, žive do 75 godina.
            \item Veličina: Srednja, krupniji su od ljudi.
            \item Brzina: brzina hoda je 30 stopa.
            \item Jezici: poluorkovi govore, pišu i čitaju zajednički i orkovski (engl. \textit{orc}).
        \end{itemize}

    \item Tiefling
        \begin{itemize}
            \item Povećanje sposobnosti: inteligencija se povećava za 1, a karizma za 2.
            \item Dob: sazrijevaju kao ljudi, žive malo duže.
            \item Veličina: Srednja, iste veličine i građe kao ljudi.
            \item Brzina: brzina hoda je 30 stopa.
            \item Jezici: tieflinzi govore, pišu i čitaju zajednički i pakleni (engl. \textit{infernal}).
        \end{itemize}
\end{itemize}

\subsection{Klase}
\label{section:klase}

Klasa je temelj onoga što likovi mogu učiniti. Ona je više od pukog zanimanja;
ona je životni poziv lika. Klasa lika oblikuje način na koji igrači doživljavaju svijet
i kako s njim komuniciraju. Klasa daje raznoliko posebne značajke, poput borčeve vještine
s oružjem i oklopom ili magovim čarolijama. Na nižim razinama, klasa nudi tek dvije ili
tri značajke, no kako lik napreduje, stječe nove, a postojeće se često poboljšavaju 
\cite{playershandbook2014}.
U tablici ispod se nalaze nazivi klasa, opisi, kockice za životne bodove koje svaka klasa
koristi kod određivanja životnih bodova po razini, primarne sposobnosti klase,
sposobnosti koje se koriste kod bacanja za spas, te stručnosti s oklopom i oružjem.
Osim ovih sposobnosti svaka klasa ima specifične značajke no kroz njih nećemo prolaziti
u ovom radu i dostupne su u \texttt{Player's Handbook}-u \cite{playershandbook2014}.

\begin{table}[h!]
    \centering
    \caption{Tablica klasa}
    \scriptsize
    \renewcommand{\arraystretch}{1.3}

    \begin{tabularx}{\linewidth}{|l|X|X|l|l|X|}
        \hline
        \textbf{Klasa} & \textbf{Opis} & \textbf{Vrsta kockica za životne bodove} & \textbf{Primarna sposobnost} & \textbf{Bacanja za spas} & \textbf{Stručnosti s oklopom i oružjem} \\ \hline
        Barbarin & Žestoki ratnik primitivnog podrijetla koji može ući u bojni bijes & d12 & Snaga & Snaga, izdržljivost & Laki i srednji oklopi, štitovi, jednostavno i borbeno oružje \\ \hline
        Bard & Nadahnjujući mađioničar čija moć odjekuje glazbom stvaranja & d8 & Karizma & Spretnost, karizma & Laki oklop, jednostavno oružje, ručni samostreli, dugi mačevi, rapiri, kratki mačevi \\ \hline
        Klerik & Svećenički prvak koji upravlja božanskom magijom u službi više sile & d8 & Mudrost & Mudrost, karizma & Laki i srednji oklopi, štitovi, jednostavno oružje \\ \hline
        Druid & Svećenik "Stare Vjere", upravlja moćima prirode i poprima životinjske oblike & d8 & Mudrost & Inteligencija, mudrost & Laki i srednji oklopi (nemetalni), štitovi (nemetalni), toljage, bodeži, strelice, koplja, mlatovi, štapovi, sablje, srpovi, praćke \\ \hline
        Borac & Majstor borbe, vješt s raznim oružjem i oklopima & d10 & Snaga ili spretnost & Snaga, izdržljivost & Svi oklopi, štitovi, jednostavno i borbeno oružje \\ \hline
        Redovnik & Majstor borilačkih vještina koji koristi snagu tijela za fizičko i duhovno savršenstvo & d8 & Spretnost, mudrost & Snaga, spretnost & Jednostavno oružje, kratki mačevi \\ \hline
        Paladin & Sveti ratnik vezan svetom zakletvom & d10 & Snaga, karizma & Mudrost, karizma & Svi oklopi, štitovi, jednostavno i borbeno oružje \\ \hline
        Rendžer & Ratnik koji koristi borilačku vještinu i magiju prirode za borbu protiv prijetnji & d10 & Spretnost, mudrost & Snaga, spretnost & Laki i srednji oklopi, štitovi, jednostavno i borbeno oružje \\ \hline
        Lupež & Nitkov koji koristi skrivanje i trikove kako bi svladao prepreke i neprijatelje & d8 & Spretnost & Spretnost, inteligencija & Laki oklop, jednostavno oružje, ručni samostreli, dugi mačevi, rapiri, kratki mačevi \\ \hline
        Čarobnjak & Bacač čarolija koji crpi urođenu magiju iz dara ili krvne loze & d6 & Karizma & Izdržljivost, karizma & Bodeži, strelice, praćke, štapovi, laki samostreli \\ \hline
        Vještac & Korisnik magije koja proizlazi iz pogodbe s izvanplanarnim entitetom & d8 & Karizma & Mudrost, karizma & Laki oklop, jednostavno oružje \\ \hline
        Mag & Učeni korisnik magije sposoban manipulirati strukturama stvarnosti & d6 & Inteligencija & Inteligencija, mudrost & Bodeži, strelice, praćke, štapovi, laki samostreli \\ \hline
    \end{tabularx}
    \caption*{(Izvor: Player's Handbook \cite{playershandbook2014})}
    \label{tab:klase}
\end{table}
\clearpage

\subsection{Ostale značajke}
\label{section:ostale-znacajke}

U ostale značajke lika ubrajamo ime, spol, visinu i težinu, opredjeljenje, jezike, 
osobne karakteristike (osobine, ideale, veze, mane) i pozadinu.
Značajke opisuju način na koji bi se lik mogao ponašati tijekom igre.
Značajke kao ime, spol, visinu, težinu i osobne karakteristike igrač sam
kreira kako bi pobliže opisao svog lika i njegovo ponašanje u svijetu,
dok se značajke kao što su opredjeljenje, pozadina mogu birati 
(ili igrači/DM mogu kreirati pozadine), te će u nastavku ukratko biti opisani.

Opredjeljenje u opisuje likove moralne i osobne stavove, te je kombinacija dvaju
faktora: jedan identificira moralnost (dobar, zao, neutralan), a drugi opisuje odnos
prema društvu i redu (zakonit, kaotičan, neutralan). Svaka kombinacija ovih dva faktora
ukupno definira devet mogućih kombinacija. 
Ovi devet opredjeljenja opisuju tipično ponašanje s tim opredjeljenjem, pojedinci
mogu značajno odstupati od tog tipičnog ponašanja \cite{playershandbook2014}:
\begin{itemize}
    \item Zakonito dobar (engl. \textit{Lawful good}) - 
        pojedinci će učiniti ispravnu stvar onako kako to društvo očekuje
    \item Neutralno dobar (engl. \textit{Neutral good}) -
        čine najbolje što mogu kako bi pomogli drugima u skaldu s svojim potrebama
    \item Kaotično dobar (engl. \textit{Chaotic good}) -
        djeluju kako im savjest nalaže, s malo obzira prema onome što
        drugi očekuju
    \item Zakonito neutralan (engl. \textit{Lawful neutral}) -
        pojedinci djeluju u skaldu sa zakonom, tradicijom ili osobnim kodeksima
    \item Neutralan (engl. \textit{Neutral}) -
        opredjeljenje onih koji radije izbjegavaju moralna pitanja i ne
        zauzimaju strane, čineći ono što se u tom trenutku čini najboljim
    \item Kaotično neutralan (engl. \textit{Chaotic neutral}) -
        slijede svoju volju, držeći svoju osobnu slobodu iznad svega
    \item Zakonito zao (engl. \textit{Lawful evil}) -
        uzimaju ono što žele, unutar granica kodeksa tradicije, lojalnosti ili reda
    \item Neutralno zao (engl. \textit{Neutral evil}) -
        opredjeljenje onih koji čine sve što im može proći nekažnjeno,
        bez suosjećanja ili grižnje savjesti
    \item Kaotično zao (engl. \textit{Chaotic evil}) -
        djeluju s proizvoljnim nasiljem, potaknuti pohlepom,
        mržnjom ili željom za krvlju
\end{itemize}

Svaka priča ima neki početak, te pozadina lika otkriva odakle lik dolazi,
kako su postali pustolov i koje je njihovo mjesto u svijetu. Odabir pozadine
pruža važne smjernice o identitetu lika, te daje liku dodatne značajke kao
stručnost u nekoj vještini ili znanje nekog jezika. U samom primjeru izrade lika
će biti detaljno opisana jedna pozadina, te sve dodatne značajke koje lik dobiva
i opremu s kojom počne.
Kratki popis pozadina \cite{playershandbook2014}:
\begin{itemize}
    \item Akolit (engl. \textit{Acolyte}) -
        osoba koja je provela život u službi hrama određenog boga,
        djelujući kao posrednik između svetog i smrtnog svijeta
    \item Šarlatan (engl. \textit{Charlatan}) -
        osoba koja zna manipulirati ljudima i koristi trikove, prevare i lažne
        identitete za vlastitu korist
    \item Kriminalac (engl. \textit{Criminal}) -
        iskusni prijestupnik s poviješću kršenja zakona i s kontaktima u podzemlju
    \item Zabavljač (engl. \textit{Entertainer}) -
        osoba koja živi za nastup pred publikom i može ih zabaviti, očarati ili
        inspirirati
    \item Narodni heroj (engl. \textit{Folk hero}) -
        osoba iz skromnog društvenog položaja, kojeg ljudi smatraju svojim
        prvakom protiv tiranije i čudovišta
    \item Pustinjak (engl. \textit{Hermit}) - 
        osoba koja živi u samoći, pronalazeći tišinu i odgovore daleko od društva
    \item Plemić (engl. \textit{Noble}) -
        osoba koja nosi plemićku titulu, posjeduje zemlju ili ima politički utjecaj i
        razumije bogatstvo i moć
    \item Mudrac (engl. \textit{Sage}) -
        osoba koja je provela godine učeći o znanju svijeta i proučavajući rukopise
    \item Mornar (engl. \textit{Sailor}) -
        osoba koja je godinama plovila na morskom plovilu, suočavajući se s olujama
        i čudovišta
    \item Vojnik (engl. \textit{Soldier}) -
        osoba kojoj je rat bio život, obučena u borbi i preživljavanju na bojnom polju
    \item Uličar (engl. \textit{Urchin}) -
        osoba koja je odrasla sama na ulici, siromašna i bez roditelja, preživljavajući
        zahvaljujući svojoj snalažljivosti
\end{itemize}

\section{Primjer lika}
\label{section:primjer-lika}

U ovom poglavlju ćemo proći kroz kreaciju lika po pravilima definiranim u prijašnjim
poglavljima. Lik će početi na prvoj razini.

Kreacija počinje s odabirom rase. Za rasu odabiremo poluorka, što znači da će lik imati
povećanu snagu i izdržljivost, snagu za dva boda, izdržljivost za jedan,
brzinu od 30 stopa, te dobiva osobine vid u mraku (engl. \textit{darkvision}) i 
prijeteću prisutnost (engl. \textit{menacing}). Prijeteća prisutnost omogućava
stručnost u vještini zastrašivanja/prijetnje (engl. \textit{intimidation}).
Još dodatne dvije sposobnosti koje daje poluork rasa su nemilosrdna
izdržljivost (engl. \textit{relentless endurance}); kad je liku broj životnih
bodova smanjen na nulu, nije odmah ubijen, već pada na jedan životni bod i ne
može opet koristiti ovu sposobnost dok ne završi dugi odmor. 
Druga je divljački napadi (engl. \textit{savage attacks}); kad lik postigne
kritični pogodak napadom oružja u bliskoj borbi, može bacati još jednu
dodatnu kockicu od kockica štete oružja i dodati vrijednost šteti kritičnog pogotka.

Za klasu odabiremo barbarina, što znači da će lik koristiti kockicu D12 za akcije
vezane uz njegove životne bodove, npr. kod prelaska na višu razinu, ili kod kratkog/dužeg odmora.
Lik dobiva stručnosti s lakim i srednjim oklopima, štitovima, jednostavnim i borbenim oružjima.
Kod bacanja spasa koristi snagu i izdržljivost, te odabira stručnosti u dvije vještine iz područja
rukovanja životinjama (engl. \textit{animal handling}), atletika (engl. \textit{athletics}),
zastrašivanja, prirode (engl. \textit{nature}), percepcije (engl. \textit{perception}) i
preživljavanja (engl. \textit{survival}).
Za ovog lika odabiremo stručnost u vještinama rukovanja životinjama i percepcije, te je lik
također vješt u zastrašivanju zbog odabrane rase.
Jedinstvena sposobnost barbarina je bijes. Tokom bitke barbarin može pobjesniti 
korištenjem bonus akcije. Dok bjesni dobiva dodatne pogodnosti, ako ne nosi teški oklop;
ima prednost na provjere snage i bacanja za spašavanje snage;
kad napada oružjem za blisku borbu, koje koristi snagu, dobiva bonu na bacanje štete;
ima otpornost na neke vrste štete, otpornost na tučenje (engl. \textit{bludgeoning}),
na probijanje (engl. \textit{piercing}) i na sječenje (engl. \textit{slashing}).
Zadnja sposobnost koju lik dobiva od klase s prvom razinom je
neoklopljena obrana (engl. \textit{armored defense}); dok lik ne nosi oklop,
klasa oklopa (AC) je jednak 10 + modifikator spretnosti + modifikator izdržljivosti
\cite{playershandbook2014}.
Na višim razinama klase lik dobiva još više mogućnosti no ovdje nećemo proći
kroz njih.

Slijedeće je potrebno odrediti vrijednost i modifikator
svake sposobnosti i maksimum životnih bodova.
Postoji više načina na koji se mogu odrediti početne vrijednosti za 
sposobnosti, no u \textit{Player's Handbook-u} je preporučeno da se 
šest puta bace četiri D6 kockice, te se kod svakog bacanja izbaci najmanja
kockica. Zbroj triju kockica određuje vrijednost jedne od sposobnosti, te nakon
šest bacanja igrač može proizvoljno rasporediti vrijednosti po sposobnostima.
Modifikatori za sposobnosti se isčitavaju iz slijedeće tablice \cite{playershandbook2014}:

\begin{table}[h!]
    \centering
    \caption{Vrijednosti sposobnosti i modifikatori}
    \begin{tabularx}{0.66\linewidth}{|X|X|X|X|}
        \hline
        \textbf{Vrijednost} & \textbf{Modifikator} & \textbf{Vrijednost} & \textbf{Modifikator} \\ \hline
        1 & -5 & 16-17 & +3 \\ \hline
        2-3 & -4 & 18-19 & +4 \\ \hline
        4-5 & -3 & 20-21 & +5 \\ \hline
        6-7 & -2 & 22-23 & +6 \\ \hline
        8-9 & -1 & 24-25 & +7 \\ \hline
        10-11 & 0 & 26-27 & +8 \\ \hline
        12-13 & +1 & 28-29 & +9 \\ \hline
        14-15 & +2 & 30 & +10 \\ \hline
    \end{tabularx}
    \\[10pt]
    \caption*{(Izvor: Player's Handbook \cite{playershandbook2014})}
    \label{tab:vrijednost-sposobnosti-i-modifikatori}
\end{table}

Maksimalni broj životnih bodova na prvoj razini određuje vrsta klase, pošto je ovaj 
lik barbarin, broj životnih bodova će iznositi 12 + modifikator izdržljivosti.
Ispod se nalazi slika s primjerom popunjenog character sheet-a za ovog lika,
neki dijelovi su popunjeni, a nisu ovdje opisani, pošto su manje važni za razumijevanje
igre i direktno se isčitavaju iz pravila (npr. oprema koju lik dobiva).

\begin{figure}[H]
    \centering
    \includegraphics[width=1\textwidth]{slike/vom_half-orc-p1.jpg}
    \caption{Primjer popunjenog character sheet-a}
    \label{fig:character-vom-half-orc}
\end{figure}

\section{Borba}
\label{section:borba}

Ovo poglavlje pruža pravila koja su potrebna kako bi likovi i čudovišta sudjelovali u borbi,
bilo to da se radi o kratkom okršaju ili produženom sukobu. Kroz cijelo poglavlje
pravila se odnose na igrača ili DM-a. DM kontrolira sva čudovišta i likove koji nisu
igrači, a uključeni su u borbu, drugi igrači kontroliraju svoje likove.
Primjer borbe će biti prikazan u poglavlju \ref{section:primjer-borbe}.

Tipičan sukob je između dvije strane i organiziran je u ciklus rundi i poteza.
Runda predstavlja oko šest sekundi u svijetu igre. Tijekom runde, svaki sudionik
u bitci ima svoj potez. Redoslijed poteza određuje se na početku borbenog susreta,
kada svi bacaju inicijativu. Nakon što svi odrade potez, borba se nastavlja u
slijedeću rundu ako nijedna strana nije pobijedila drugu.

Prije početka borbe DM određuje tko bi mogao biti iznenađen na početku borbe.
Ako nijedna strana ne pokušava biti prikrivena automatski primjećuju jedna drugu
i ovaj korak se preskače, no ako se jedna strana uspješno prikrade ostvaruje prednost.
Strana koje je iznenađena, tj. lik ili čudovište koje je iznenađeno se ne može pomaknuti ili 
poduzeti akciju u svom prvom potezu borbe, također ne može poduzeti reakciju dok
taj prvi potez ne završi. Član grupe može biti iznenađen čak i ako ostali članovi nisu \cite{playershandbook2014}.

Inicijativa određuje redoslijed poteza tokom borbe. Kada borba počne svaki sudionik radi 
provjeru spretnosti (engl. \textit{dexterity check}) kako bi odredio svoje mjesto u poretku inicijative.
Baca D20 kockicu i na dobivenu vrijednost dodaje modifikator spretnosti.
DM određuje inicijativu za sve likove, čudovišta koje ne kontroliraju drugi igrači.
Nakon dobivenih vrijednosti za inicijativu, DM ih rangira od najviše do najmanje
vrijednosti te određuje redoslijed poteza. U slučajevima kad igrači imaju istu
inicijativu mogu međusobno odrediti tko nastupa prvu, ako čudovište i igrač imaju
jednaku inicijativu DM određuje redoslijed.

\subsection{Igračev potez}
\label{section:igracev-potez}

Kad je igrač na potezu može pomicati svog lika do udaljenosti
određenom brzinom lika i odraditi jednu akciju. U pravilu svako polje je veličine
pet sa pet stopa, što bi značilo da se lik sa brzinom od trideset
stopa može pomaknuti maksimalno šest polja.
Igrač odlučuje ako će se prvo kretati ili odraditi akciju.
U poglavlju \ref{section:akcije} će biti opisane najčešće akcije koje likovi mogu
odraditi, mnoge značajke različitih klasa i druge sposobnosti pružaju dodatne opcije za akcije.

Razne značajke klase, čarolije i druge sposobnosti omogućavaju poduzimanje dodatne akcije u 
trenutnom potezu. Dodatna akcije se može jedino poduzeti u slučaju kad značajka klase,
čarolija ili neka druga sposobnost navodi da se nešto može odraditi kao dodatna akcija.
Po potezu, jednako kao s akcijom, može se odraditi samo jedna dodatna akcija, te ako
lik ima dostupno više dodatnih akcija potrebno je odrediti koju je najbolje iskoristiti u tom
slučaju \cite[str. 189]{playershandbook2014}.

Kao dio poteza lik može stupiti u interakciju s jednim predmetom ili značajkom
okoline, ta interakcija se može dogoditi tijekom kretanja ili akcije.
Na primjer, otvaranje vrata dok se lik kreće prema neprijatelju ili izvlačenje
oružja kao dio iste akcije za napad bi spadale pod interakciju s okolinom.

Određene posebne sposobnosti, čarolije i situacije omogućuju poduzimanje posebne akcije
zvane reakcija. Reakcija je trenutni odgovor na okidač neke vrste, koji se može dogoditi
tokom poteza igrača ili tuđeg poteza. Napad prilike (engl. \textit{Opportunity attack}) je 
najčešća vrsta reakcije i više je opisana u poglavlju \ref{section:napadi}.

\subsection{Kretanje i položaj}
\label{section:kretanje-i-polozaj}

Kad je lik na potezu može se pomaknuti do udaljenosti definirane brzinom lika.
Po potezu lik može potrošiti proizvoljnu količinu brzine, ne mora potrošiti svu
ili uopće iskoristiti brzinu.
Kretanje se može razdijeliti kroz cijeli potez. Moguće je koristiti dio brzine prije
i poslije akcije. Na primjer, ako je brzina lika trideset stopa, može se kretati deset stopa,
iskoristiti neku akciju, a zatim se kretati još dvadeset stopa.
Ako akcija uključuje više od jednog napada oružjem, kretanje se može dodatno razdvojiti
između tih napada.

Borbe se rijetko odvijaju u praznim sobama. Špilje pune kamenja i guste šume
su tipična mjesta odvijanja borbe i sadrže težak teren (engl. \textit{difficult terrain}).
Za svako kretanje kroz ovakav teren potrebno je utrošiti jednu dodatnu stopu za svaku stopu
koju lik želi preći. Na primjer, ako se radi o udaljenosti od pet stopa, za prolaz kroz
težak teren potrebno je utrošiti deset stopa brzine \cite[str. 190]{playershandbook2014}.

Likovi se mogu kretati kroz prostor prijateljskih stvorenja, ali ne mogu završiti potez
ili zaustaviti se na istom polju. Kroz prostor neprijateljskih stvorenja mogu se kretati samo
ako je stvorenje barem dvije veličine veće ili manje od lika koji se kreće, no nije dozvoljeno
završavanje kretanja u prostoru koji zauzima to stvorenje \cite[str. 191]{playershandbook2014}.

U tablici ispod se nalaze veličine stvorenja i različite količine prostora koje zauzimaju,
tokom borbe veća stvorenja kontroliraju veću količinu prostora, također objekti u okolini ponekad
mogu koristiti kategorije veličine. 
Prostor stvorenja je područje u stopama koje to stvorenje kontrolira tokom borbe, a nije
izraz njihovih fizičkih dimenzija.

\begin{table}[H]
    \centering
    \caption{Kategorije veličine}
    \begin{tabularx}{0.66\linewidth}{|X|X|X|X|}
        \hline
        \textbf{Veličina} & \textbf{Prostor} \\ \hline
        Sićušan & 2.5 sa 2.5 stopa \\ \hline
        Mala & 5 sa 5 stopa \\ \hline
        Srednja & 5 sa 5 stopa \\ \hline
        Velika & 10 sa 10 stop \\ \hline
        Ogromna & 15 sa 15 stopa \\ \hline
        Gigantska & 20 sa 20 stopa ili više \\ \hline
    \end{tabularx}
    \\[10pt]
    \caption*{(Izvor: Player's Handbook \cite[str. 191]{playershandbook2014})}
    \label{tab:kategorije-velicine}
\end{table}

\subsection{Akcije}
\label{section:akcije}

Kad je lik na potezu i želi odraditi neku akciju, osim akcija koje dobiva od klase ili
nekih drugih posebnih značajki uvijek može odraditi neku od slijedećih akcija 
ili improvizirati akciju. Ako je akcija improvizirana DM odlučuje ako je moguća,
koju vještinu lik može iskoristiti za akciju i kakvo bacanje kockice će odrediti
uspješnost akcije.

\textbf{Napad} je najčešća akcija u borbi, bilo da se zamahuje mačem,
ispaljuje strijela iz luka ili tučnjava šakama, ovom akcijom izvodi se jedan napad
izbliza ili napad na daljinu \cite[str. 192]{playershandbook2014}.
Napadi će biti detaljnije opisani u poglavlju \ref{section:napadi}.

\textbf{Bacanje čarolija}, mnoge vrste čudovišta i likova imaju pristup čarolijama i mogu iv
koristiti u borbi. Svaka čarolija ima definirano vrijeme bacanja (engl. \textit{cast time}),
koje određuje mora li bacač koristiti akciju, reakciju, minute ili sate vremena da baci čaroliju.
U nekim slučajevima, kao kad čarolija ima vrijeme bacanja od više sati, bacanje čarolije nije
nužno jedna akcija. Većina čarolija ima vrijeme bacanja od jedne akcije, pa je često takva čarolija
korištena tokom borbe \cite[str. 192]{playershandbook2014}.

\textbf{Trk} (engl. \textit{dash}), omogućava liku da potroši akciju kako bi se mogao
na istom potezu dodatno kretati. Povećanje je jednako brzini lika, nakon što se primjene
bilo kakvi modifikatori brzine. Na primjer, s brzinom od trideset stopa, na istom potezu
moguće je pomaknuti se šestdeset stopa ako se iskoristi ova akcija.
Bilo kakve promjene na brzinu također imaju utjecaj na dodatnu brzinu dobivenu ovom akcijom,
na primjer ako je lik usporen i ima smanjenu brzinu s trideset na petnaest, korištenjem ove akcije
će dobiti dodatnih petnaest brzine \cite[str. 192]{playershandbook2014}.

\textbf{Isključivanje iz borbe} (engl. \textit{disengage}), akcija omogućuje kretanje bez da
neprijatelji dobivaju napad prilike kad se lik kreće oko njih \cite[str. 192]{playershandbook2014}.

\textbf{Izmicanje} (engl. \textit{dodge}), akcija omogućava usredotočenost na izbjegavanje napada.
Do početka slijedećeg poteza, svako bacanje napada protiv lika koji je iskoristio ovu akciju,
ima odsutnost prednosti ako lik vidi napadača, a sva bacanja za spas
na spretnost se izvode s prednošću \cite[str. 192]{playershandbook2014}.

\textbf{Pomoć}, likovi mogu pružiti pomoć drugom liku u izvršavanju nekog zadatka.
Korištenjem ove akcije drugi lik dobiva prednost na slijedeću provjeru sposobnosti koju
izvrši za obavljanje zadatka u kojem dobiva pomoć, pod uvjetom da napravi tu radnju prije
nego pomagač opet dođe na potez. Također moguće je pomoći prijateljskom stvorenju u napadu
na neprijatelja unutar pet stopa, napadač na svom potezu dobiva prednost na prvo bacanje za 
napad \cite[str. 192]{playershandbook2014}.

\textbf{Pretraga}, akcija može biti iskorištena za pretragu okoline i pronalaženju nečega.
Ovisno o prirodi pretrage, DM može zatražiti da se napravi provjera vještine percepcija ili 
istraga \cite[str. 193]{playershandbook2014}.

\subsection{Napadi}
\label{section:napadi}

Napade možemo podijeliti u nekoliko vrsta, napadi na blizinu, na daljinu i
napadi prilike. Svaka vrsta napada se ima razlike no svi imaju jednaku
strukturu i način na koji se izvode \cite[str. 193-194]{playershandbook2014}:
\begin{enumerate}
    \item Odabir mete; potrebno je odabrati metu unutar dometa napada,
        meta može biti stvorenje, predmet ili lokacija. Na primjer,
        lokacija se može odabrati ako pokušavamo napasti nevidljivog
        neprijatelja ili ako čarolija koju bacamo ima utjecaj na veće
        područje.
    \item Određivanje modifikatora; DM određuje ako se meta skriva iza zaklona
        i ako napadač ima prednost ili nema prednost protiv mete. Uz to,
        čarolije, posebne sposobnosti i drugi učinci mogu imati utjecaja.
    \item Razriješivanje napada; igrač radi bacanje za napad (engl. \textit{attack roll}),
        ovo bacanje određuje ako je meta uspješno pogođena napadom ili je napad promašio.
        U slučaju pogotka određuje se šteta koja je napravljena meti, također bacanjem kockice,
        ovisno o napadu koriste se različite kockica, neki napadi uzrokuju posebne učinke
        uz štetu ili umjesto nje.
\end{enumerate}

Za izvođenje bacanja za napad koristi se d20 kockica te se na dobivenu vrijednost dodaju
određeni modifikatori. Ako je ukupan zbroj bacanja i modifikatora jednak ili veći od klase
oklopa mete, u tom slučaju napad pogađa. Dva najčešća modifikatora koji se dodaju
su modifikator sposobnosti (engl. \textit{ability modifier}) i bonus stručnosti lika.
Modifikator sposobnosti odgovara sposobnosti koja se koristi za napad, npr. može biti snaga
ili spretnost ovisno o vrsti oružja. Bonus stručnosti se dodaje ako lik ima stručnost u korištenju
oružja kojim napada.

U rjeđim slučajevima kao rezultat bacanja d20 moguće je dobiti 1 ili 20. Ako je
dobivena vrijednost 20, napad je uvijek pogodak, neovisno o bilo kakvim modifikatorima
ili klasi oklopa mete, također ovakav napad je uvijek kritičan pogodak.
Ako je dobivena vrijednost 1, napad uvijek promašuje i ne dodaje se modifikator ni
ne provjerava klasa oklopa mete.
Slična funkcija je kod ostalih bacanja d20, vrijednost od 20 je skoro uvijek uspjeh
dok je vrijednost 1 neuspjeh.

\subsection{Šteta i iscjeljivanje}
\label{section:steta-i-iscjeljivanje}

Ozljede i rizik od smrti stalni su pratitelji onih koji istražuju svjetove D\&D-a.
Životni bodovi predstavljaju kombinaciju fizičke i mentalne izdržljivosti.
Stvorenje s više životnih bodova teže je ubiti ili onesvijestiti. Ona s manje
životnih bodova su krhkija. Trenutni životni bodovi stvorenju mogu biti bilo koji
broj između nule i maksimuma životnih bodova.
Kad kod stvorenje pretrpi štetu, ta šteta oduzima od životnih bodova,
gubitak životnih bodova nema utjecaj na sposobnosti stvorenja sve dok je
vrijednost veća od nula.

Ako šteta ne rezultira smrću stvorenja, tada nije trajna. Likovi mogu
vratiti životne bodove odmorom, korištenjem čarolija ili ispijanjem ljekovitog
napitka. Kada stvorenje primi iscjeljivanje bilo koje vrste, vraćeni životni bodovi
dodaju se trenutnim životnim bodovima i ne mogu premašiti maksimum životnih bodova
tog lika ili stvorenja.

Postoji nekoliko situacije koje se mogu dogoditi kad neki lik ili stvorenje padne na 
nula životnih bodova. Manje važna stvorenja, čudovišta uglavnom umiru odmah
nakon što im životni bodovi padnu na nulu, dok važniji likovi mogu pasti u
nesvijest ili imati nekoliko pokušaja da se spase od smrti i stabiliziraju
svoje stanje.
U slučaju kad količina štete svede lika na nula životnih bodova i ostatak
štete je veći ili jednak broju maksimalnih životnih bodova lika, lik umire u tom
trenutku.
Padom u nesvijest tijekom borbe znači da lik ili stvorenje na svom potezu
mora izvršiti bacanje za spas od smrti (engl. \textit{death save}).
Za spas se koristi d20 i sve vrijednosti jednake ili veće od deset
približavaju lika životu, dok sve ispod deset smrti.
Na treći uspjeh lik postaje stabilan, te se više ne mora spašavati od smrti,
dok na treći neuspjeh lik umire. Bacanje dvadeset ili jedan se broji kao
dva uspijeha ili neuspjeha.
Stabilizirano stvorenje ima nula životnih bodova i onesviješteno je, odmorom
se životni bodovi mogu povratiti, no odmor traje neko vrijeme i nije moguć
tokom borbe. Tokom borbe iscjeljivanje na barem jedan životni bod
je jedini način za osvijestiti lika ili stvorenje.

\chapter{Prikaz implementacije}
\label{chapter:prikaz-implementacije}

U ovom poglavlju će biti opisani alati i tehnologije korištene za implementaciju platforme.
Korišteno je više različitih tehnologija, fokus je na odabir i korištenje tehnologija 
koje su što više prikladne za implementacije web aplikacija.
Ovime se osigurava da je aplikacija dostupna, funkcionalna i prenosiva
preko više različitih operativnih sustava i web preglednika.

Platforma je implementirana na način da može biti korištena kao zasebna web
aplikacija koja se spaja na udaljeni poslužitelj ili kao desktop aplikacija
koja automatski u pozadini pokreće poslužitelj i omogućava korisniku 
da koristi aplikaciju bez posebno pokretanja i postavljanja poslužitelja.
Desktop i mobilna integracija je moguća kroz korištenje tauri programskog okvira.

Tauri je programski okvir za izgradnju malih, brzih programa za sve desktop i mobilne platforme.
Aplikacija može integrirati bilo koji frontend okvir koji se kompajlira u HTML, javascript, css
i koristiti ove tehnologije za razvoj korisničkog sučelja, a istovremeno može koristiti jezike
poput rusta, swifta, jave i kotlina za backend logiku i interakciju s hardverom \cite{tauridocs}.
U ovom projektu tauri služi za pakiranje projekta u jednu izvršnu datoteku, te jednostavno
postavljanje i pokretanje poslužitelja i jednog klijenta.

Aplikacija se sastoji od dva odvojena dijela, to su poslužitelj i klijent.
Poslužitelj služi za spremanje podataka, spajanje i sinkronizaciju podataka između
povezanih klijenata, te učitavanje slika, kako bi bile dostupne klijentima preko web API-a,
te kreiranje manjih sličica (engl. \textit{thumbnail}) iz učitanih slika, da klijenti ne bi
trebali preuzimati cijele slike koje mogu biti veće rezolucije, a nikada neće biti potrebna
takva rezolucija. Klijent sadrži svo korisničko sučelje i logiku igre, komunicira kroz
poslužitelja s ostalim klijentima te prima promjene o stanju igre od poslužitelja.

Cijela aplikacija je umrežena, sve promjene koje se događaju kod jednog klijenta budu
sinkronizirane i prikazane kod ostalih klijenata koji su povezani u istu sesiju.
Aplikacija koristi ravnopravne čvorove (engl. \textit{peer-to-peer})
za umrežavanje zbog jednostavnosti implementacije i
veće fleksibilnosti, prije implementacije u obzir je došlo umrežavanje s autoritativnim poslužiteljem,
više o oba pristupa i zašto je odabrano peer-to-peer umrežavanje u poglavlju \ref{section:mrezna-komunikacija}.

\section{Web aplikacije}
\label{section:web-aplikacije}

Web aplikacija je program klijent-poslužitelj arhitekture koji koristi
web preglednik kao svog klijenta. Web aplikacija obavljanje interaktivnu
uslugu povezivanja s poslužiteljem putem interneta ili intraneta \cite{shklar2009web}.

Razlikuje se od tradicionalne web stranice na slijedeće načine \cite{shklar2009web}:
\begin{itemize}
    \item \textbf{Dinamičko naspram statičkog}: dok web stranica, u većini slučajeva
        isporučuje sadržaj statičkih datoteka, web aplikacija prikazuje dinamički prilagođen
        sadržaj na temelju parametara zahtjeva, korisničke sesije.
    \item \textbf{Interaktivnost}: web aplikacije omogućuju korisnicima obavljanje specifičnih
        zadataka, poput kupnje robe ili upravljanje zalihama.
\end{itemize}

Web aplikacije se temelje na distribuiranoj arhitekturi koja se sastoji od tri glavna elementa.
Prvi element je klijent još nekad i nazivan frontend; to je obično web preglednik instaliran
na korisnikovom računalu. Preglednik je odgovoran za prikazivanje sadržaja i rukovanje interakcijama
korisnika.
Drugi element je poslužitelj, ponekad zvan backend; ovaj element sluša dolazne zahtjeve od
klijenata i generira odgovore. Obrađuje zadatke poput posluživanja statičkih datoteka,
autentifikacije i prosljeđivanja dinamičkih zahtjeva aplikacijskim programima.
Treći element je baza podataka; web aplikacije često zahtijevaju pristup trajnim podacima.
Povezuju se s bazama podataka kako bi pohranile i dohvaćale informacije.
Arhitektura aplikacije uključuje specifičnu logiku za omogućavanje pristupa podacima i
upravljanje transakcijama \cite{casteleyn2009engineering}.

\section{Poslužitelj}
\label{section:posluzitelj}

U ovom potpoglavlju prolazimo kroz implementaciju poslužitelja, tehnologije koje su
korištene i odluke koje su donesene vezane uz dizajn cijele aplikacije, kao što je
odluka za implementaciju peer-to-peer umrežavanja umjesto autoritativnog poslužitelja.

Poslužitelj je implementiran u programskom jeziku rust. Rust je sistemski programski jezik
sa statičnim tipovima podataka, dizajniran za performanse i sigurnost. 
Razvijen od strane Mozilla Research, stvoren je kao odgovor na dugogodišnju napetost između
kontrole na niskoj razini koju pruže jezici poput C i C++-a i sigurnosti jezika više razine
poput jave ili pythona. Rust sigurnost postiže sustavom zvan vlasništvo (engl. \textit{ownership}) i 
provjeravanjem vlasništva nad varijablama tokom kompajliranja programa sustavom zvanim provjeravač
posuđenog (engl. \textit{borrow checker}). Korištenjem ova dva sustava osigurava korektan pristup
i dijeljenje memorije, izbjegava probleme s korištenjem izbrisane memorije, čitanja ili
pisanje u memoriju koju više dretva istovremeno koristi, a pristup memoriji nije sinkroniziran \cite{rustdocs}. 

Budući da rust nema sakupljača smeća (engl. \textit{garbage collector}) i ima minimalan runtime,
nevjerojatno je brz i memorijski učinkovit. Kod kompajliranja programa koristi apstrakcije bez troška, što
znači da programer može koristiti koncepte programiranje više razine, poput iteratora ili generičnih funkcija,
bez da računalo obavlja dodatni posao tijekom izvođenja, svi koncepti više razine tokom kompajliranja su 
pretvoreni u nižu razinu, na primjer iteratori su pretvoreni u jednostavne for petlje, dok su generične funkcije
pretvorene u više različitih funkcija koje kompajlirani program poziva \cite{rustdocs}.

\subsection{Protokoli za prijenos podataka između poslužitelja i klijenta}
\label{section:protokoli}

Implementirani poslužitelj koristi http i websockete za komunikaciju s klijentima.
HTTP je protokol na kojim se temelji world wide web, dizajniran je za prijenos specijaliziranih
poruka putem mreže. Koristi TCP kao svoj temeljni transportni sloj, što osigurava da su svi podaci
preneseni u točnom redoslijedu i primljeni. Komunikacija http-a koristi model zahtjeva i odgovora.
HTTP klijent poput web preglednika započinje komunikaciju slanje poruke zahtjeva prema http
poslužitelju. Poslužitelj zatim zaprima zahtjev, obrađuje ga i generira poruku odgovora koju šalje
natrag klijentu. Na ovaj način klijent može zatražiti bilo kakvu vrstu resursa od poslužitelja \cite{casteleyn2009engineering}.

HTTP zahtjev se sastoji od nekoliko dijelova i počinje s metodom zahtjeva, putanju do resursa i broj verzije http-a,
slijedeći redovi su popis zaglavlja, nakon zaglavlja je prazni redak iza kojeg se u poruku dodaje tijelo.
Tijelo je opcionalno i ne mora biti prisutno u poruci. Ispod je primjer zahtjeva, korištena je metoda GET, pokušava se dohvatiti
resurs index.html, verzija HTTP-a je 1.1 i u zahtjev uključeno je jedno zaglavlje, tijelo je prazno.
Ovakvi zahtjevi su uglavnom slani s klijenta prema poslužitelju, no nije neubičajeno da dva poslužitelja komuniciraju
preko http-a, te u ovom slučaju jedan poslužitelj preuzima ulogu klijenta.

\begin{verbatim}
GET /index.html HTTP/1.1
User-Agent: Mozilla/5.0(X11;Linux x86_64) Chrome/143.0.0.0
\end{verbatim}

Klijent od poslužitelja prima odgovor na zahtjev. Struktura odgovora počinje statusnom linijom,
koja sadrži verziju http-a, troznamenkasti statusni kod i kratko objašnjenje poslanog koda, npr. OK.
U odgovoru, jednako kao kod zahtjeva, nakon prve linije dolazi lista zaglavlja koja su poslana, te
nakon zaglavlja prazni redak i tijelo odgovora. U slučaju ispod tijelo je popunjeno i klijent prima
html dokument koji može prikazati u web pregledniku.

\begin{verbatim}
HTTP/1.1 200 OK
Content-Type: text/html
Content-Lenght: 40

<html><h1>Hello World!</h1></html>
\end{verbatim}

HTTP protokol je protokol bez stanja, što znači da je životni vijek veze između klijenta 
i poslužitelja ograničen na jednu razmjenu zahtjeva i odgovora. Poslužitelj ne održava stanje
veze tijekom prijenosa uzastopnih naredbi. Ne može pamtiti niz interakcija niti grupirati
zahtjeve zajedno. Budući da protokol ne održava informacije o sesiji, web aplikacije
moraju koristiti druge mehanizme, poput kolačića za uspostavljanje trajnih sesija ili stanja
kroz višestruke zahtjeve \cite{casteleyn2009engineering}.

Websocket je protokol koji omogućuje dvosmjernu komunikaciju između klijenta i poslužitelja
preko jedne, dugotrajne TCP veze. Za razliku od klasičnog http-a koji koristi model
zahtjev odgovor, websocket omogućuje da obje strane šalju poruke u bilo kojem trenutku, bez
potrebe za novim zahtjevima \cite{pimentel2012communicating}.
Ova karakteristika je idealna za aplikacije koje zahtijevaju ažuriranja u stvarnom vremenu i možemo
nazvati websocket kao protokol koji održava stanje, za razliku od http-a koji je bez stanja.
Uspostava websocket veze počinje kao običan http zahtjev koji je nadograđen na websocket protokol.
Klijent šalje posebni http GET zahtjev s zaglavljem \texttt{Upgrade: websocket}. 
Poslužitelj odgovara statusom \texttt{101 Switching Protocols} ako prihvaća zahtjev.
Nakon ovoga, veza prelazi na websocket protokol. Poruke se šalju u obliku okvira i
postoji nekoliko vrsta okvira, to su tekst okviri, za slanje tekstualnih poruka,
binarni okviri, za slanje biranih poruka i kontrolni okviri, za slanje keep-alive
poruka i zatvaranje veze \cite{gupta2018overview}.
U kasnijim poglavljima su prikazani primjeri gdje i kako se oba protokola koriste.

\subsection{Axum}
\label{section:axum}

\subsection{Socketioxie}
\label{section:socketioxie}

\subsection{Sqlite}
\label{section:sqlite}

\subsection{Mrežna komunikacija}
\label{section:mrezna-komunikacija}

\subsection{Endpoints}
\label{section:endpoints}

Naivna implementacija mrežne komunikacije između klijenata

\subsection{Mrežne tehnologije u web preglednicima}
\label{section:mrezne-tehnologije-u-web-preglednicima}


\section{Klijent}

\subsection{Peer to peer komunikacija}
\subsection{Stateless \& stateful komunikacija}

\chapter{Korišteni alati i tehnologije za implementaciju}

Aplikacija je podijeljena na dva dijela, na poslužitelja i klijenta.
Poslužitelj se koristi za:
\begin{itemize}
    \item kreiranje igara,
    \item povezivanje klijenata; sustav je napravljen s umrežavanjem klijenata,
        više klijenata mogu igrati jednu igru i pratiti događanja,
    \item spremanje i učitavanje podataka o aktivnim igrama;
        promjene koje klijenti naprave kroz igru moraju biti spremljene
        za kasnije učitavanje i sinkronizaciju ostalih klijenata,
\end{itemize}

Opisati svaki alat koji se koristi.
Navesti u kratko gdje se koristi.
Navesti da budu točni primjeri kasnije u radu.

\section{Poslužitelj}

\subsection{Tauri}
\subsection{Axum}

\section{Baza podataka}

\section{Klijent}

\subsection{Pixi.js}
\subsection{React.js}
\subsection{Jotai}
\subsection{Socket.io}
\subsection{Mantine}

\section{Primjer borbe}
\label{section:primjer-borbe}

Ovo je glavni dio rada u kojem treba razraditi temu, pojasniti istraživanja, prikazati rezultate i slično. Poželjno je na početku poglavlja dati kratki opis strukture poglavlja, kako bi čitatelj/čitateljica rada mogao/mogla lakše pratiti složenu cjelinu.

\section{Poglavlje druge razine}

\subsection{Poglavlje treće razine}

\subsubsection{Poglavlje četvrte razine}

\chapter{Prikaz slučajeva korištenja}

Tehničke upute u nastavku opisuju način tehničkog oblikovanja rada i navođenja literature.

\section{Upute za oblikovanje izgleda rada}

\begin{flushleft}\textbf{Stranice} se oblikuju korištenjem sljedećih parametara:\end{flushleft}

\begin{itemize}
    \item veličina i oblik papira je A4, okomito usmjerenje, margine 2,5 cm na svakoj strani;

    \item naslovna stranica rada se ne numerira;

    \item nakon naslovne stranice, sve sljedeće stranice do 1. Poglavlja se numeriraju rimskim brojevima, počevši od i;

    \item od 1. poglavlja nadalje, stranice se numeriraju arapskim brojevima;

    \item broj stranice treba pozicionirati desno 1,25 cm od dna stranice, font Arial 9.
\end{itemize}
\begin{flushleft}\textbf{Tekst} rada je potrebno oblikovati sukladno ovom predlošku, odnosno na sljedeći način:\end{flushleft}
\begin{itemize}
    \item u pisanju teksta koristite font Arial 11 pt, s proredom 1,5 te razmakom 0 pt prije i razmakom 6 pt poslije odlomka, pri čemu je prvi redak uvučen za 1,25 cm;

    \item u naslovima prve razine „3. Razrada teme“ koristite font Arial 18 pt, podebljano, prijelom stranice (svaki naslov prve razine treba biti na novoj stranici), s proredom 1,5 te razmakom 0 pt prije i razmakom 18 pt poslije odlomka;

    \item u naslovima druge razine „2.1. Naslov“ koristite font Arial 16 pt, podebljano, s proredom 1,5 te razmakom 18 pt prije i razmakom 12 pt poslije odlomka;

    \item u naslovima treće razine „2.1.1. Naslov“ koristite font Arial 14 pt, podebljano, s proredom 1,5 te razmakom 12 pt prije i razmakom 6 pt poslije odlomka;

    \item u naslovima četvrte razine „2.1.1.1. Naslov“ koristite font Arial 12 pt, podebljano, s proredom 1,5 te razmakom 6 pt prije i razmakom 6 pt poslije odlomka;

    \item ostalo značajno isticanje cjelina rada može biti istaknuto podebljanim i kurziv slovima, korištenjem fonta Arial 11 pt.
\end{itemize}


\begin{flushleft}\textbf{Slike} u radu je potrebno oblikovati na sljedeći način:
naziv slike navedite ispod slike uz numeraciju;\end{flushleft}

\begin{itemize}
    \item za nazive slika koristite iste postavke fonta kao i za tekst, ali stavite naziv slike u centrirani položaj;

    \item za oblikovanje same slike koristite font Arial 9 pt za tekst na slici;
ispred same slike umetnite jedan prazan redak (osim ako je slika pozicionirana na početku stranice);

    \item nakon naziva slike ostavite jedan redak prazan (osim ako je naziv slike zadnji redak na stranici);

    \item kod prijeloma stranice treba obratiti posebnu pozornost da naziv slike, izvor i sama slika moraju biti na istoj stranici; 

    \item slike je potrebno numerirati redom pojavljivanja u tekstu;

    \item ako je slika preuzeta iz drugog izvora, nakon navođenja naziva slike u zagradi navedite izvor, npr. (autor/autorica, godina);

    \item dozvoljeno je napraviti vlastitu preradu slika, grafikona ili tablica na način da se zadrži isti smisao sadržaja, ali promijeni izgled. I u takvim se slučajevima obavezno u nazivu navodi referenca izvornog djela ovako: “(Prema: Klačmer Čalopa i Cingula, 2012)“;

    \item dozvoljeno je preuzeti samo jednu sliku, grafikon ili tablicu u izvornom obliku iz istog izvora. Za doslovno preuzimanje većeg dijela sadržaja potrebno je ishoditi dozvolu nositelja autorskih prava;

    \item primjer označavanja slike možete vidjeti u nastavku (slika \ref{fig:podjela}).
\end{itemize}

\begin{figure}[h!]
    \centering
    \includegraphics[width=0.9\textwidth]{slike/slika.png}
    \caption{Podjela investicijskih fondova (Izvor: \citeauthor{Aranda2009}, \citeyear{Aranda2009})}
    \label{fig:podjela}
\end{figure}

\begin{flushleft}\textbf{Tablice} rada je potrebno oblikovati sukladno ovim uputama:\end{flushleft}
\begin{itemize}
    \item naziv tablice navedite iznad slike;

    \item za nazive tablica koristite iste postavke fonta kao i za tekst, ali stavite naziv tablice u centrirani položaj;

    \item za oblikovanje same tablice koristite font Arial 9 pt za tekst u tablici;

    \item tablice numerirajte redom pojavljivanja u tekstu;

    \item prije naziva tablice umetnite jedan redak prazan (osim ako je naziv tablice prvi redak na stranici);

    \item nakon same tablice umetnite jedan prazan redak (osim ako je tablica pozicionirana na kraju stranice);

    \item kod prijeloma stranice treba obratiti posebnu pozornost da naziv tablice, izvor i sama tablica moraju biti na istoj stranici; 

    \item ako je tablica preuzeta iz drugog izvora, nakon navođenja naziva tablice potrebno je navesti izvor, na isti način kako je opisano kod slika;

    \item primjer označavanja tablice možete vidjeti u nastavku (tablica \ref{tab:objekti}).
\end{itemize}

\begin{table}[h!] 
    \centering
    \caption{Prikaz podataka o učestalosti pojavljivanja objekta}
    \begin{tabularx}{0.66\textwidth}{|X|X|X|X|}
        \hline
         \cellcolor{gray!25} & \cellcolor{gray!25} & \cellcolor{gray!25} & \cellcolor{gray!25} \\
        \hline
         &  &  &  \\
        \hline
         &  &  & \\
        \hline
    \end{tabularx}
    \\[10pt]
    \caption*{(Izvor: \citeauthor{caragliu2011smart}, \citeyear{caragliu2011smart})}
    \label{tab:objekti}
\end{table}

\begin{flushleft}\textbf{Programski kod}\end{flushleft}
\begin{itemize}
    \item za oblikovanje teksta koji je programski kôd koristite font Courier, veličine 10 pt, jednostruki prored, 6 pt iza odlomka, npr. HTML kôd dijela zaglavlja početne web stranice FOI weba:
\end{itemize}

\begin{lstlisting}[language=HTML]
<head>
  <meta http-equiv="Content-Type" content="text/html; charset=utf-8" />
  <link rel="shortcut icon" href="https://www.foi.unizg.hr/sites/default/files/favicon_0_1.ico" type="image/vnd.microsoft.icon" />
  <meta name="generator" content="Drupal 7 (http://drupal.org)" />
  <link rel="canonical" href="https://www.foi.unizg.hr/hr" />
  <link rel="shortlink" href="https://www.foi.unizg.hr/hr" />
  <!-- Set the viewport width to device width for mobile -->
  <meta name="viewport" content="width=device-width, initial-scale=1.0">
  <title>Dobro %*došli*) na FOI | FOI</title>...
</head>
\end{lstlisting}

\begin{flushleft}\textbf{Formule}\end{flushleft}
\begin{itemize}
    \item za unos formula koristite editor za formule u svom tekst procesoru.
\end{itemize}

\begin{flushleft}\textbf{Kratice}\end{flushleft}   
\begin{itemize}
    \item ako želite koristiti kratice pojmova u tekstu, kad prvi put spominjete pojam potrebno je navesti puni naziv, a kraticu navesti u zagradi (npr. Informacijske i komunikacijske tehnologije, kraće IKT). Nakon toga možete koristiti kratice u tekstu. Poželjno je u naslovima koristiti pune nazive.
\end{itemize}

\begin{flushleft}\textbf{Strano nazivlje}\end{flushleft}   
\begin{itemize}
    \item strano nazivlje se u tekstu navodi u zagradi, napisano \textit{kurzivom}, nakon hrvatskog izraza, npr. Analiza društvene mreže (engl. \textit{Social Network Analysis - SNA}).
\end{itemize}

\section{Navođenje literature}

Za navođenje literature u radu možete odabrati i koristiti jedan od sljedeća dva ponuđena stila: \textbf{APA} ili \textbf{IEEE} stil. Važno je samo dosljedno primjenjivati odabrani stil u cijelom radu.

U popisu literature potrebno je navesti svu literaturu i samo literaturu koju ste koristili u tekstu.

Uz svaku preuzetu tvrdnju potrebno je navesti njezin izvor, tj. referencu. Reference se u tekstu navode tako da se uz citirani tekst navede izvor, sukladno načinu propisanom odabranim stilom i FOI preporukama za citiranje i referenciranje \cite{SchattenEtAl2016roadmap}.

\chapter{Zaključak}

Ovdje treba sažeto rezimirati najvažnije rezultate razrade teme rada. Potrebno je sažeto opisati što je predmet rada, koje su metode, tehnike, programski alati ili aplikacije korištene u razradi rada te koje su pretpostavke dokazane, a koje opovrgnute. Sadržajno, ono što se u uvodu rada najavljuje i kasnije je obuhvaćeno u samom radu, moralo bi biti opisano u zaključnom dijelu kroz rezultate rada. 

\lipsum[1-2]

\printbibliography[title=Popis literature]
\addcontentsline{toc}{chapter}{Popis literature}

\listoffigures
\addcontentsline{toc}{chapter}{Popis slika}
 
\listoftables
\addcontentsline{toc}{chapter}{Popis popis tablica}

\appendix
\renewcommand{\thechapter}{\arabic{chapter}}

\end{document}
