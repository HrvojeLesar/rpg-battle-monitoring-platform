\documentclass{foi}
\usepackage[utf8]{inputenc}
\usepackage{lipsum}

\vrstaRada{\diplomski} % \diplomski
\title{Izrada platforme za praćenje bitke u igri igranja uloga}

\author{Hrvoje Lesar}
\spolStudenta{\musko} % \zensko ili \musko
\mentor{Markus Schatten}
\spolMentora{\musko} % \zensko ili \musko
\godina{2025}
\mjesec{rujan}
\date{2025}
%\status{redoviti}
\indeks{0016133479}
\smjer{Organizacija poslovnih sustava} % (ili Poslovni sustavi, Ekonomika poduzetništva, Primjena informacijske tehnologije u poslovanju, Informacijsko i programsko inženjerstvo, Baze podataka i baze znanja, Organizacija poslovnih sustava, Informatika u obrazovanju)
\titulaProfesora{Prof. dr. sc.}

\sazetak{Opsega od 100 do 300 riječi. Sažetak upućuje na temu rada, ukratko se iznosi čime se rad bavi, teorijsko-metodološka polazišta, glavne teze i smjer rada te zaključci.}

\kljucneRijeci{riječ; riječ; ...riječ; Obuhvaća 7+/-2 ključna pojma koji su glavni predmet rasprave u radu.}

\begin{document}

\maketitle

\tableofcontents

\pagestyle{plain}

\chapter{Uvod}
\label{chapter:uvod}

\begin{itemize}
    \item Rpg
    \item Cilj
    \item Prvi dio...
    \item Praktični dio
\end{itemize}

\chapter{Igre igranja uloga}
\label{chapter:igra-igranja-uloga}

Igre igranja uloga predstavljaju jedan od najsloženijih i kreativnih spojeva
priče, izvedbe i igre u suvremenoj kulturi. 
Ovakva igra poziva igrače da zajednički stvaraju i nastanjuju zamišljeni svijet,        
vođeni kombinacijom strukturiranih pravila i spontanog pripovijedanja.
Svaka sesija je istovremeno i igra i priča, a njezin ishod oblikuju
mašta igrača i slučajnost ishoda bacanja kockica.

Igre igranja uloga nemaju jednu definiciju i možemo ih definirati s različitih gledišta:
\begin{itemize}
    \item Situacija igranja uloga definirana kao situacija u kojoj se od pojedinca izričito traži da preuzme
        ulogu koja inače nije njegova, ili ako jest njegova, onda u okruženju koje nije
        uobičajeno za izvođenje te uloge. \cite{mann1956experimental}
    \item Igranje uloga nije jedinstvena jasno definirana aktivnost, već čitav niz aktivnosti
        okupljenih pod pogodnim nazivom. Na jednom kraju spektra nalazi se intenzivno odražavanje
        sobnih emocija, dok se na drugom kraju nalazi situacija u kojoj je preuzimanje uloge
        bliže konceptu zagovaranja. \cite{white2024tabletop}
    \item Igranje uloga je umjetnost iskustva, a stvaranje igra uloga znači kreiranje novih iskustava. \cite{white2024tabletop}
    \item Igranje uloga definira se kao bilo koji čin u kojem se istovremeno stvara, dodaje i promatra imaginarna stvarnost. \cite{white2024tabletop}
    \item Igra igranja uloga mora se sastojati od interaktivnog pripovijedanja: 
        sposobnosti likova i razrješenja radnji, definirani su brojevima ili količinama, kojima se
        manipulira prema određenim pravilima. Donošenje odluka igrača pokreće i pomiče priču u naprijed.
        Uz skupinu koja djeluje kao autor, priča organski raste i odigrava se, bivajući doživljena
        od svojih stvaratelja. \cite{white2024tabletop}
\end{itemize}

\section{Ratne igre}
\label{section:ratne-igre}

Početak i inspiracija za igre igranja uloga dolazi iz ratnih igara.
Nasljeđe ratnih igara seže od šaha, budući da su prve igre unutar posebne
kategorije ratnih igara uvelike posuđivale ploče, figure i mehanike upravo
iz šaha. Georg von Rei{\ss}witz se smarta ocem ratnih igara, jer je razvio
prvi sustav ratnih igara koji je široko korišten ako ozbiljan alat za obuku
i istraživanje. Razvijenu igru su nazvali tzv. \texttt{Das Kriegsspiel} te je igra
zadovoljila dugo prepoznatu potrebu za jeftinim i lako ponovljivim sredstvom
za obuku časnika u zapovijedanju i planiranju bitke. Vojne ratne igre gotovo se uvijek
fokusiranju na sadašnjost, na ondašnje vojske, tehnologiju i države u vrijeme
njihova nastanka. Veliki dio razvoja ratnih igara osamnaestog
i devetnaestog stoljeća održava neprestana poboljšanja u sredstvima i 
provedbi ratovanja te posljedičnu potrebu da se ratne igre stalno usklađuju s
realnošću. \cite{peterson2012playing}

Tek kad su se hobisti počeli poigravati ovim sustavima, uspjeli su osloboditi
ratne igre ograničenja suvremenog konteksta i istražiti povijesna razdoblja,
buduće moguće svjetove, pa čak i nemoguće fantastične svjetove.
Hobisti su također odbacili strogo reproduciranje stvarnih uvjeta na
bojištu u korist više uravnoteženijeg pristupa koji je kombinirao
realističnost i igrivost. Do 1960-ih, ovi zaigrani ljubitelji ratnih igara
transformirali su ih iz sredstva za vojnu obuku u znatno maštovitiju aktivnost,
onu koja je mogla poslužiti kao temelj za modeliranje događaja u igri
poput Dungeons \& Dragons (skraćeno D\&D). \cite{peterson2012playing}

\section{Nastanak sustava Dungeons \& Dragons}
\label{section:nastanak-sustava-dungeons-dragons}

Dungeons \& Dragons (kraće poznato kao D\&D), igra nastala 1974.
godine predstavlja ključni trenutak u povijesti igra uloga.
Igru su razvili Gary Gygax i Dave Arneson, koji su težili proširenju
mogućnosti stolnih ratnih igara prema novim narativnim i
imaginativnim domenama. Prije D\&D-a, Gygax je već imao utjecaj
u wargaming zajednici, osobito kroz svoj rad na igri \texttt{Chainmail},
srednjovjekovnoj miniaturskoj ratnoj igri koju je razvio i bio jedan
od autora 1971. godine. 
\texttt{Chainmail} je izvorno zamišljen kao skup pravila za simulaciju
srednjevjekovnih bitaka s minijaturama, no uključivao je i fantastični dodatak
koji je omogućava igračima da u igru unesu mistična bića i magijske elemente.
Ova dopuna održavala je Gygaxovu fascinaciju srednjovjekovnom literaturom i
fantastičnim narativima, posebno djelima J.R.R Tolkeina i 
Roberta E. Howarda. \cite{sidhu2024fifty}

Dave Arneson je eksperimentirao s narativnim pristupom u svojoj kampanji
Blackmoor, u kojoj su se pojedinačni likovi mogli razvijati kroz
više sesija i aktivno utjecati na tijek priče, za razliku od
tradicionalnog wargaminga u koje igrači kontroliraju cijele jedinice ili vojske.
Arnesonove inovacije, u kombinaciji s Gygaxovim strukturiranim sustavom pravila
iz Chainmaila, stvorile su temelje za mehaniku i narativni potencijal igre
D\&D. \cite{sidhu2024fifty}

Prvo izdanje D\&D-a je izdano 1974. godine, bilo je u početku skromno 
po opsegu, no uvelo je revolucionarne koncepte;
igrači su mogli preuzeti uloge različitih likova istraživati fantastične tamnice,
sudjelovati u zadacima vođenim od strane Dungeon Mastera (DM-a),
koji moderira svijet igre.
Pravila D\&D-a uključivala su elemente miniaturskog wargaminga,
vjerojatnosnih mehanika s kockama i suradničkog pripovijedanja,
stvarajući novi oblik interaktivne zabave koji je spajao strategiju,
maštu i narativ. Tijekom vremena igra se razvila kroz više izdanja,
od kojih je svako unaprijedilo pravila i proširilo mogućnosti, no osnovni
princip suradničkog pripovijedanja i igre vođene likovima je ostao nepromijenjen.

\chapter{Opis pravila igre - Dungeons \& Dragons}
\label{chapter:opis-pravila-igre-dungeons-dragons}

U ovome poglavlju će biti opisana pravila igre D\&D-a.
Pravila igre su se razvila od početne koncepcije igre 70-tih godina 20. stoljeća.
Poglavlje će biti usredotočeno na pravila definirana u \texttt{Player's Handbook 2014}.
Svakim novim izdanjem igre pravila se suptilno nagorađuju, poboljšavaju i razvijaju, te se 
dodaju nove mehanike igre. Ovdje definiramo zadnje izdanje koje će se poglavlje i kasnija
implementacije koristiti i pratiti.

Dungeons \& Dragons je igra igranja uloga posvećena pripovijedanju u svjetovima
mača i magije. Dijeli elemente s dječjim igrama pretvaranja. Kao i te igre, D\&D pokreće
mašta. Radi se o zamišljanju visokog dvorca pod olujnim noćnim nebom i predočavanju kako bi
pustolov mogao reagirati na izazove koje taj prizor predstavlja.
Za razliku od igra pretvaranja, D\&D pričama daje strukturu, način određivanja posljedica
za akcije igrača. Igrači bacaju kockice kako bi razriješili jesu li njihovi napadi pogodili
ili promašili, ili mogu li se njihovi pustolovi popeti na liticu, izmaknuti se udaru čarobne munje
ili izvesti neki drugi opasan pothvat. Sve je moguće, ali kockice čine neke ishode vjerojatnijima
od drugih. \cite{playershandbook2014}

Jedan igrač preuzima ulogu Dungeon Mastera, glavnog pripovjedača i suca igre.
DM stvara pustolovine za likove, koji se kreću kroz njihove opasnosti i odlučuje koje
će putove istražiti. Više o DM-u će biti u potpoglavlju \ref{section:dungeon-master}.

Svaki igrač stvara pustolova (koji se naziva i likom) i udružuje se s drugim
pustolovima. Radeći zajedno, grupa može istraživati mračnu tamnicu, ukleti dvorac,
izgubljeni hram... Pustolovi mogu rješavati zagonetke, razgovarati s drugim
likovima, boriti se protiv čudovišta i otkrivati čudesne čarobne predmete i blago.

\section{Lista termina}
\label{section:lista-termina}

Ovo poglavlje će služiti za definiranje terminologije i kratica koje će se
koristiti kroz rad. Svaki termin će biti kraće opisan i potencijalno imati
engleski naziv (koji je uglavnom više prepoznatljiv nego hrvatski prijevod),
no cijeli kontekst ne mora biti razumljiv u spisu, već će biti opisan u nekom od sljedećih poglavlja.

\begin{itemize}
    \item \textbf{D\&D} Dungeons \& Dragons
    \item \textbf{DM} Dungeon Master
    \item \textbf{NPC} Non-Player Character
    \item \textbf{D20} Kockica s dvadeset strana
    \item \textbf{D2, D4, D6, D8, D10, D12} Kockice s određenim brojem strana
    \item \textbf{AC} Armour class, težina zadatka
    \item \textbf{DC} Difficulty class, klasa oklopa
\end{itemize}

\section{Dungeon Master}
\label{section:dungeon-master}

Dungeon master je kreativna sila iza D\&D sesije. DM stvara svijet koji ostali igrači
istražuju, te također osmišljava i vodi pustolovine koje pokreću priču.
Pustolovina se obično temelji na uspješnom dovršetku neke potrage i može trajati
kratko poput jedne sesije ili se protezati kroz više sesija.
Duže pustolovine mogu uvući igrače u velike sukobe za čije je rješavanje potrebno
više sesija. Kada se povežu u niz pustolovine čine trajnu priču nazvanu kampanja.
D\&D kampanja može uključivati desetke pustolovina i trajati jednu sesiju, mjesecima ili
pa ćak i godinama. \cite{dungeonmastersguide2014}

Dungeon master preuzima mnogo različitih uloga. Kao arhitekt kampanje, DM stvara pustolovine
postavljajući pozicije čudovišta, zamka i blaga koje likovi drugih igrača mogu otkriti.
Kao pripovjedač, DM pomaže ostalim igračima vizualizirati što se događa oko njih,
improvizirajući kada pustolovi učine nešto ili odu negdje neočekivano. Kao glumac, DM
igra uloge čudovišta i sporednih likova, time im daje život.
A kao sudac, DM tumači pravila i odlučuje kada ih se treba pridržavati, a kada ih promijeniti.
Svaki DM pristupa ulogama izmišljanja, pisanju, pripovijedanju, improvizaciji, glumi i suđenju
drugačije i vjerojatno će igrači i sam DM u nekim uživati više nego u drugima.

D\&D pravila pomažu DM-u i ostalim igračima da se dobro zabave, ali pravila nisu glavni autoritet.
DM je glavni i upravlja igrom te smije "kršiti" ili mijenjati definirana pravila. 
Ipak, cilj DM-a nije pobiti pustolove, već stvoriti svijet kampanje koji se vrti oko njihovih
djela i odluka, te osigurati da se igrači vraćaju po još.

Uspjev D\&D sesije ovisi o sposobnosti DM-a da zabavi ostale igrače za stolom.
Dok je njihova uloga stvaranje likova, udahnuti im život i pomoći usmjeravati kampanju
kroz postupke svojih likova. DM treba održati igrače (i sebe) zainteresiranima i uživljenima
u svijet koji je stvorio te omogućiti njihovim likovima da čine nevjerojatne stvari.
Poznavanje u kojim dijelovima D\&D-a igrači najviše uživaju pomaže DM-u stvoriti i voditi
avanture u kojima će uživati i kojih će se sjećati. \cite{dungeonmastersguide2014}

Aktivnosti na koje se DM može fokusirati za bolje zadovoljstvo igrača \cite{dungeonmastersguide2014}:
\begin{itemize}
    \item Gluma; igrači koji uživaju u glumi vole ulaziti u ulogu svog lika i govoriti njihovim klasom.
        Kao roleplayeri u duši, oni uživaju u društvenim interakcijama s NPC-ovima, čudovištima i ostalim
        članovima družine.
    \item Istraživanje; igrači koji žude za istraživanjem žele iskusiti čuda koja nudi fantastični svijet.
        Žele znati što se nalazi iza sljedećeg ugla ili brda, također vole pronalaziti skrivene tragove i blago
    \item Poticanje akcije; neki igrači vole poticati akciju, željni su da se stvari događaju čak
        i ako to znači preuzimanje opasnih rizika. Rađe će srljati u opasnost i suočiti se s posljedicama
        nego se suočiti s dosadom.
    \item Borba; drugi igrači uživaju u fantastičnoj borbi poput premlaćivanja zlikovaca i čudovišta.
        Traže bilo kakav izgovor da započne borba, preferirajući akciju nad pažljivim promišljanjem.
    \item Optimizacija lika; igrači koji uživaju u optimizaciji sposobnosti svojih likova vole
        podešavati svoje likove za vrhunske borbene performanse stjecanjem razina, novih značajki i
        čarobnih predmeta, rado će prihvatiti svaku priliku da pokažu nadmoć svojih likova
    \item Rješavanje problema; igrači koji žele rješavati probleme vole proučavati motivacije
        drugih likova, razmrsiti spletke zlikovaca, rješavati zagonetke i smišljati planove
    \item Pripovijedanje; igrači koji vole pripovijedanje žele doprinijeti priči. Sviđa im se kada su njihovi
        likovi uključeni u priču koja se razvija i uživaju u susretima koji su vezani s
        glavnom radnjom i proširuju je.
\end{itemize}

\section{Sustav D20}
\label{section:sustav-d20}

Dvadeseto strana kockica je najčešće korištena za određivanje uspjeha ili neuspjeha
radnji u D\&D-u i tu kockicu nazivom d20. Ponekad se koriste dvanaesto strane, osmo
strane, šesto strane, četvero strane kockice i novčić, one također imaju nazive
ovisno o broju strana, d12, d10, d8, d6, d4 i d2.

Svi likovi i čudovišta u igri definirani su kroz šest osnovnih sposobnosti: 
snaga, spremnost, izdržljivost, inteligencija, mudrost i karizma.
Kod većine pustolova vrijednosti tih sposobnosti kreću se od 3 do 18,
dok kod čudovišta mogu biti niske poput 1 ili visoke do 30. 
Te vrijednosti, kao i modifikatori koji iz njih proizlaze, temelj su gotovo
svakoh bacanja d20 kockice

Tri glavne vrste bacanja čine srž pravila igre: provjera sposobnosti,
bacanje za napad i bacanje za spas. Sve tri radnje slijede nekoliko koraka:
\begin{enumerate}
    \item bacanje kockice i dodavanje modifikatora. Na rezultat bačene d20 kockice dodajemo odgovarajući modifikator.
    Najčešće je to modifikator jedne od šest sposobnosti, a ponekad uključuje i bonus stručnosti koji
    odražava specifičnu vještinu lika.
    \item uračunavanje dodatnih bonusa i kazni. Značajke klase, čarolije, specifične okolnosti ili neki drugi
    efekti mogu dodati bonus na rezultat ili ga umanjiti.
    \item usporedba rezultata s ciljanim brojem. Ako je ukupan zbroj jednak ili veći od ciljnog broj, radnja
    (provjera, napad, spašavanje) je uspješno, u suprotnom, nije uspjela. DM obično određuje ciljne brojeve
    i govori igračima jesi li uspjeli odraditi radnju. DM ne treba otkriti igračima ciljani broj za uspjeh.
\end{enumerate}

Ciljni broj za provjere sposobnosti i spas naziva se težina zadatka ili difficulty class (DC), dok se ciljni
broj za napad naziva klasa oklopa ili Armour Class (AC). Ovo jednostavno pravilo temelj je 
rješavanja većine situacija u D\&D-u.

\section{Kreacija lika}
\label{section:kreacija-lika}

U ovom poglavlju će biti prikazana kreacija lika, odabir rase i
klasa i ukratko prikazati pojedinosti koje likovi dobivaju odabirom.
Na slici ispod se nalazi primjer praznog lista o liku (tzv. character sheet).
Character sheet koristi igračima za zapisivanje i praćenje karakteristika
njihovih likova. Kako ćemo prolaziti kroz različite dijelove kreacije lika
odgovarajući podaci na listu će biti popunjeni, da bi kroz primjer prošli
kroz kreaciju lika.

\begin{figure}[H]
    \centering
    \includegraphics[width=0.7\textwidth]{slike/character_sheet.png}
    \caption{Character sheet \cite{charactershet2014}}
    \label{fig:character-sheet}
\end{figure}

% \begin{figure}[!htb]
%     \centering
%     \includegraphics[width=0.9\textwidth]{slike/character_sheet.png}
%     \caption{Character sheet \cite{charactershet2014}}
%     \label{fig:character-sheet}
% \end{figure}

\subsection{Odabir rase}
\label{section:rase}

\subsection{Odabir klase}
\label{section:klase}

\section{Avanture}
\label{section:avanture}

\subsection{Vrijeme i prolaz vremena}
\label{section:vrijeme-i-prolaz-vremena}

\subsection{Brzina}
\label{section:brzina}

\subsection{Odmor}
\label{section:odmor}

\section{Bitka}
\label{section:bitka}

\chapter{Prikaz implementacije}

\chapter{Struktura programa}

Razdvojeno na klijent, poslužitelj strukturu
Koristi se peer to peer komunikacija, postoji server authoritive...

\section{Poslužitelj}

\section{Klijent}

\section{Mrežna komunikacija}

Naivna implementacija mrežne komunikacije između klijenata

\subsection{Peer to peer komunikacija}
\subsection{Stateless \& stateful komunikacija}

\chapter{Korišteni alati i tehnologije za implementaciju}

U ovom poglavlju će biti opisani alati i tehnologije korištene za implementaciju platforme.
Korišteno je više različitih tehnologija, fokus je na tehnologije koje imaju
fokus je na korištenje tehnologija koje su što više prikladne za implementacije 
aplikacije kao web aplikaciju.
Ovime se osigurava da je aplikacija dostupna, funkcionalna i prenosiva
preko više različitih operativnih sustava i web preglednika.

Aplikacija je podijeljena na dva dijela, na poslužitelja i klijenta.
Poslužitelj se koristi za:
\begin{itemize}
    \item kreiranje igara,
    \item povezivanje klijenata; sustav je napravljen s umrežavanjem klijenata,
        više klijenata mogu igrati jednu igru i pratiti događanja,
    \item spremanje i učitavanje podataka o aktivnim igrama;
        promjene koje klijenti naprave kroz igru moraju biti spremljene
        za kasnije učitavanje i sinkronizaciju ostalih klijenata,
\end{itemize}

Opisati svaki alat koji se koristi.
Navesti u kratko gdje se koristi.
Navesti da budu točni primjeri kasnije u radu.

\section{Poslužitelj}

\subsection{Tauri}
\subsection{Axum}
\subsection{Socketioxie}

\section{Baza podataka}

\section{Klijent}

\subsection{Pixi.js}
\subsection{React.js}
\subsection{Jotai}
\subsection{Socket.io}
\subsection{Mantine}

Ovo je glavni dio rada u kojem treba razraditi temu, pojasniti istraživanja, prikazati rezultate i slično. Poželjno je na početku poglavlja dati kratki opis strukture poglavlja, kako bi čitatelj/čitateljica rada mogao/mogla lakše pratiti složenu cjelinu.

\section{Poglavlje druge razine}

\subsection{Poglavlje treće razine}

\subsubsection{Poglavlje četvrte razine}

\chapter{Prikaz slučajeva korištenja}

Tehničke upute u nastavku opisuju način tehničkog oblikovanja rada i navođenja literature.

\section{Upute za oblikovanje izgleda rada}

\begin{flushleft}\textbf{Stranice} se oblikuju korištenjem sljedećih parametara:\end{flushleft}

\begin{itemize}
    \item veličina i oblik papira je A4, okomito usmjerenje, margine 2,5 cm na svakoj strani;

    \item naslovna stranica rada se ne numerira;

    \item nakon naslovne stranice, sve sljedeće stranice do 1. Poglavlja se numeriraju rimskim brojevima, počevši od i;

    \item od 1. poglavlja nadalje, stranice se numeriraju arapskim brojevima;

    \item broj stranice treba pozicionirati desno 1,25 cm od dna stranice, font Arial 9.
\end{itemize}
\begin{flushleft}\textbf{Tekst} rada je potrebno oblikovati sukladno ovom predlošku, odnosno na sljedeći način:\end{flushleft}
\begin{itemize}
    \item u pisanju teksta koristite font Arial 11 pt, s proredom 1,5 te razmakom 0 pt prije i razmakom 6 pt poslije odlomka, pri čemu je prvi redak uvučen za 1,25 cm;

    \item u naslovima prve razine „3. Razrada teme“ koristite font Arial 18 pt, podebljano, prijelom stranice (svaki naslov prve razine treba biti na novoj stranici), s proredom 1,5 te razmakom 0 pt prije i razmakom 18 pt poslije odlomka;

    \item u naslovima druge razine „2.1. Naslov“ koristite font Arial 16 pt, podebljano, s proredom 1,5 te razmakom 18 pt prije i razmakom 12 pt poslije odlomka;

    \item u naslovima treće razine „2.1.1. Naslov“ koristite font Arial 14 pt, podebljano, s proredom 1,5 te razmakom 12 pt prije i razmakom 6 pt poslije odlomka;

    \item u naslovima četvrte razine „2.1.1.1. Naslov“ koristite font Arial 12 pt, podebljano, s proredom 1,5 te razmakom 6 pt prije i razmakom 6 pt poslije odlomka;

    \item ostalo značajno isticanje cjelina rada može biti istaknuto podebljanim i kurziv slovima, korištenjem fonta Arial 11 pt.
\end{itemize}


\begin{flushleft}\textbf{Slike} u radu je potrebno oblikovati na sljedeći način:
naziv slike navedite ispod slike uz numeraciju;\end{flushleft}

\begin{itemize}
    \item za nazive slika koristite iste postavke fonta kao i za tekst, ali stavite naziv slike u centrirani položaj;

    \item za oblikovanje same slike koristite font Arial 9 pt za tekst na slici;
ispred same slike umetnite jedan prazan redak (osim ako je slika pozicionirana na početku stranice);

    \item nakon naziva slike ostavite jedan redak prazan (osim ako je naziv slike zadnji redak na stranici);

    \item kod prijeloma stranice treba obratiti posebnu pozornost da naziv slike, izvor i sama slika moraju biti na istoj stranici; 

    \item slike je potrebno numerirati redom pojavljivanja u tekstu;

    \item ako je slika preuzeta iz drugog izvora, nakon navođenja naziva slike u zagradi navedite izvor, npr. (autor/autorica, godina);

    \item dozvoljeno je napraviti vlastitu preradu slika, grafikona ili tablica na način da se zadrži isti smisao sadržaja, ali promijeni izgled. I u takvim se slučajevima obavezno u nazivu navodi referenca izvornog djela ovako: “(Prema: Klačmer Čalopa i Cingula, 2012)“;

    \item dozvoljeno je preuzeti samo jednu sliku, grafikon ili tablicu u izvornom obliku iz istog izvora. Za doslovno preuzimanje većeg dijela sadržaja potrebno je ishoditi dozvolu nositelja autorskih prava;

    \item primjer označavanja slike možete vidjeti u nastavku (slika \ref{fig:podjela}).
\end{itemize}

\begin{figure}[h!]
    \centering
    \includegraphics[width=0.9\textwidth]{slike/slika.png}
    \caption{Podjela investicijskih fondova (Izvor: \citeauthor{Aranda2009}, \citeyear{Aranda2009})}
    \label{fig:podjela}
\end{figure}

\begin{flushleft}\textbf{Tablice} rada je potrebno oblikovati sukladno ovim uputama:\end{flushleft}
\begin{itemize}
    \item naziv tablice navedite iznad slike;

    \item za nazive tablica koristite iste postavke fonta kao i za tekst, ali stavite naziv tablice u centrirani položaj;

    \item za oblikovanje same tablice koristite font Arial 9 pt za tekst u tablici;

    \item tablice numerirajte redom pojavljivanja u tekstu;

    \item prije naziva tablice umetnite jedan redak prazan (osim ako je naziv tablice prvi redak na stranici);

    \item nakon same tablice umetnite jedan prazan redak (osim ako je tablica pozicionirana na kraju stranice);

    \item kod prijeloma stranice treba obratiti posebnu pozornost da naziv tablice, izvor i sama tablica moraju biti na istoj stranici; 

    \item ako je tablica preuzeta iz drugog izvora, nakon navođenja naziva tablice potrebno je navesti izvor, na isti način kako je opisano kod slika;

    \item primjer označavanja tablice možete vidjeti u nastavku (tablica \ref{tab:objekti}).
\end{itemize}

\begin{table}[h!] 
    \centering
    \caption{Prikaz podataka o učestalosti pojavljivanja objekta}
    \begin{tabularx}{0.66\textwidth}{|X|X|X|X|}
        \hline
         \cellcolor{gray!25} & \cellcolor{gray!25} & \cellcolor{gray!25} & \cellcolor{gray!25} \\
        \hline
         &  &  &  \\
        \hline
         &  &  & \\
        \hline
    \end{tabularx}
    \\[10pt]
    \caption*{(Izvor: \citeauthor{caragliu2011smart}, \citeyear{caragliu2011smart})}
    \label{tab:objekti}
\end{table}

\begin{flushleft}\textbf{Programski kod}\end{flushleft}
\begin{itemize}
    \item za oblikovanje teksta koji je programski kôd koristite font Courier, veličine 10 pt, jednostruki prored, 6 pt iza odlomka, npr. HTML kôd dijela zaglavlja početne web stranice FOI weba:
\end{itemize}

\begin{lstlisting}[language=HTML]
<head>
  <meta http-equiv="Content-Type" content="text/html; charset=utf-8" />
  <link rel="shortcut icon" href="https://www.foi.unizg.hr/sites/default/files/favicon_0_1.ico" type="image/vnd.microsoft.icon" />
  <meta name="generator" content="Drupal 7 (http://drupal.org)" />
  <link rel="canonical" href="https://www.foi.unizg.hr/hr" />
  <link rel="shortlink" href="https://www.foi.unizg.hr/hr" />
  <!-- Set the viewport width to device width for mobile -->
  <meta name="viewport" content="width=device-width, initial-scale=1.0">
  <title>Dobro %*došli*) na FOI | FOI</title>...
</head>
\end{lstlisting}

\begin{flushleft}\textbf{Formule}\end{flushleft}
\begin{itemize}
    \item za unos formula koristite editor za formule u svom tekst procesoru.
\end{itemize}

\begin{flushleft}\textbf{Kratice}\end{flushleft}   
\begin{itemize}
    \item ako želite koristiti kratice pojmova u tekstu, kad prvi put spominjete pojam potrebno je navesti puni naziv, a kraticu navesti u zagradi (npr. Informacijske i komunikacijske tehnologije, kraće IKT). Nakon toga možete koristiti kratice u tekstu. Poželjno je u naslovima koristiti pune nazive.
\end{itemize}

\begin{flushleft}\textbf{Strano nazivlje}\end{flushleft}   
\begin{itemize}
    \item strano nazivlje se u tekstu navodi u zagradi, napisano \textit{kurzivom}, nakon hrvatskog izraza, npr. Analiza društvene mreže (engl. \textit{Social Network Analysis - SNA}).
\end{itemize}

\section{Navođenje literature}

Za navođenje literature u radu možete odabrati i koristiti jedan od sljedeća dva ponuđena stila: \textbf{APA} ili \textbf{IEEE} stil. Važno je samo dosljedno primjenjivati odabrani stil u cijelom radu.

U popisu literature potrebno je navesti svu literaturu i samo literaturu koju ste koristili u tekstu.

Uz svaku preuzetu tvrdnju potrebno je navesti njezin izvor, tj. referencu. Reference se u tekstu navode tako da se uz citirani tekst navede izvor, sukladno načinu propisanom odabranim stilom i FOI preporukama za citiranje i referenciranje \cite{SchattenEtAl2016roadmap}.

\chapter{Zaključak}

Ovdje treba sažeto rezimirati najvažnije rezultate razrade teme rada. Potrebno je sažeto opisati što je predmet rada, koje su metode, tehnike, programski alati ili aplikacije korištene u razradi rada te koje su pretpostavke dokazane, a koje opovrgnute. Sadržajno, ono što se u uvodu rada najavljuje i kasnije je obuhvaćeno u samom radu, moralo bi biti opisano u zaključnom dijelu kroz rezultate rada. 

\lipsum[1-2]

\printbibliography[title=Popis literature]
\addcontentsline{toc}{chapter}{Popis literature}

\listoffigures
\addcontentsline{toc}{chapter}{Popis slika}
 
\listoftables
\addcontentsline{toc}{chapter}{Popis popis tablica}

\appendix
\renewcommand{\thechapter}{\arabic{chapter}}

\end{document}
