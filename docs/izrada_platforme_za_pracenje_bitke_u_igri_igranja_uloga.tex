\documentclass{foi}
\usepackage[utf8]{inputenc}
\usepackage{lipsum}

\vrstaRada{\diplomski} % \diplomski
\title{Izrada platforme za praćenje bitke u igri igranja uloga}

\author{Hrvoje Lesar}
\spolStudenta{\musko} % \zensko ili \musko
\mentor{Markus Schatten}
\spolMentora{\musko} % \zensko ili \musko
\godina{2025}
\mjesec{rujan}
\date{2025}
%\status{redoviti}
\indeks{0016133479}
\smjer{Organizacija poslovnih sustava} % (ili Poslovni sustavi, Ekonomika poduzetništva, Primjena informacijske tehnologije u poslovanju, Informacijsko i programsko inženjerstvo, Baze podataka i baze znanja, Organizacija poslovnih sustava, Informatika u obrazovanju)
\titulaProfesora{Prof. dr. sc.}

\sazetak{Opsega od 100 do 300 riječi. Sažetak upućuje na temu rada, ukratko se iznosi čime se rad bavi, teorijsko-metodološka polazišta, glavne teze i smjer rada te zaključci.}

\kljucneRijeci{riječ; riječ; ...riječ; Obuhvaća 7+/-2 ključna pojma koji su glavni predmet rasprave u radu.}

\begin{document}

\maketitle

\tableofcontents

\pagestyle{plain}

\chapter{Uvod}

\begin{itemize}
    \item Rpg
    \item Cilj
    \item Prvi dio...
    \item Praktični dio
\end{itemize}

\chapter{Igranje uloga i igre igranja uloga}

Igre igranja uloga predstavljaju jedan od najsloženijih i kreativnih spojeva
priče, izvedbe i igre u suvremenoj kulturi. 
Ovakva igra poziva igrače da zajednički stvaraju i nastanjuju zamišljeni svijet,        
vođeni kombinacijom strukturiranih pravila i spontanog pripovijedanja.
Svaka sesija je istovremeno i igra i priča, a njezin ishod oblikuju
mašta igrača i slučajnost ishoda bacanja kockica.

Igre igranja uloga nemaju jednu definiciju i možemo ih definirati s različitih gledišta:
\begin{itemize}
    \item Situacija igranja uloga definirana kao situacija u kojoj se od pojedinca izričito traži da preuzme
        ulogu koja inače nije njegova, ili ako jest njegova, onda u okruženju koje nije
        uobičajeno za izvođenje te uloge. \cite{mann1956experimental}
    \item Igranje uloga nije jedinstvena jasno definirana aktivnost, već čitav niz aktivnosti
        okupljenih pod pogodnim nazivom. Na jednom kraju spektra nalazi se intenzivno odražavanje
        sobnih emocija, dok se na drugom kraju nalazi situacija u kojoj je preuzimanje uloge
        bliže konceptu zagovaranja. \cite{white2024tabletop}
    \item Igranje uloga je umjetnost iskustva, a stvaranje igra uloga znači kreiranje novih iskustava. \cite{white2024tabletop}
    \item Igranje uloga definira se kao bilo koji čin u kojem se istovremeno stvara, dodaje i promatra imaginarna stvarnost. \cite{white2024tabletop}
    \item Igra igranja uloga mora se sastojati od interaktivnog pripovijedanja: 
        sposobnosti likova i razrješenja radnji, definirani su brojevima ili količinama, kojima se
        manipulira prema određenim pravilima. Donošenje odluka igrača pokreće i pomiče priču u naprijed.
        Uz skupinu koja djeluje kao autor, priča organski raste i odigrava se, bivajući doživljena
        od svojih stvaratelja. \cite{white2024tabletop}
\end{itemize}

\section{Ratne igre}

Početak i inspiracija za igre igranja uloga dolazi iz ratnih igara.
Nasljeđe ratnih igara seže od šaha, budući da su prve igre unutar posebne
kategorije ratnih igara uvelike posuđivale ploče, figure i mehanike upravo
iz šaha. Georg von Rei{\ss}witz se smarta ocem ratnih igara, jer je razvio
prvi sustav ratnih igara koji je široko korišten ako ozbiljan alat za obuku
i istraživanje. Razvijenu igru su nazvali tzv. \texttt{Das Kriegsspiel} te je igra
zadovoljila dugo prepoznatu potrebu za jeftinim i lako ponovljivim sredstvom
za obuku časnika u zapovijedanju i planiranju bitke. Vojne ratne igre gotovo se uvijek
fokusiranju na sadašnjost, na ondašnje vojske, tehnologiju i države u vrijeme
njihova nastanka. Veliki dio razvoja ratnih igara osamnaestog
i devetnaestog stoljeća održava neprestana poboljšanja u sredstvima i 
provedbi ratovanja te posljedičnu potrebu da se ratne igre stalno usklađuju s
realnošću. \cite{peterson2012playing}

Tek kad su se hobisti počeli poigravati ovim sustavima, uspjeli su osloboditi
ratne igre ograničenja suvremenog konteksta i istražiti povijesna razdoblja,
buduće moguće svjetove, pa čak i nemoguće fantastične svjetove.
Hobisti su također odbacili strogo reproduciranje stvarnih uvjeta na
bojištu u korist više uravnoteženijeg pristupa koji je kombinirao
realističnost i igrivost. Do 1960-ih, ovi zaigrani ljubitelji ratnih igara
transformirali su ih iz sredstva za vojnu obuku u znatno maštovitiju aktivnost,
onu koja je mogla poslužiti kao temelj za modeliranje događaja u igri
poput Dungeons \& Dragons. \cite{peterson2012playing}


\section{Nastanak sustava Dungeons \& Dragons}

\chapter{Opis pravila igre - Dungeons \& Dragons}

\chapter{Prikaz implementacije}

\chapter{Struktura programa}

Razdvojeno na klijent, poslužitelj strukturu
Koristi se peer to peer komunikacija, postoji server authoritive...

\section{Poslužitelj}

\section{Klijent}

\section{Mrežna komunikacija}

Naivna implementacija mrežne komunikacije između klijenata

\subsection{Peer to peer komunikacija}
\subsection{Stateless \& stateful komunikacija}

\chapter{Korišteni alati i tehnologije za implementaciju}

U ovom poglavlju će biti opisani alati i tehnologije korištene za implementaciju platforme.
Korišteno je više različitih tehnologija, fokus je na tehnologije koje imaju
fokus je na korištenje tehnologija koje su što više prikladne za implementacije 
aplikacije kao web aplikaciju.
Ovime se osigurava da je aplikacija dostupna, funkcionalna i prenosiva
preko više različitih operativnih sustava i web preglednika.

Aplikacija je podijeljena na dva dijela, na poslužitelja i klijenta.
Poslužitelj se koristi za:
\begin{itemize}
    \item kreiranje igara,
    \item povezivanje klijenata; sustav je napravljen s umrežavanjem klijenata,
        više klijenata mogu igrati jednu igru i pratiti događanja,
    \item spremanje i učitavanje podataka o aktivnim igrama;
        promjene koje klijenti naprave kroz igru moraju biti spremljene
        za kasnije učitavanje i sinkronizaciju ostalih klijenata,
\end{itemize}

Opisati svaki alat koji se koristi.
Navesti u kratko gdje se koristi.
Navesti da budu točni primjeri kasnije u radu.

\section{Poslužitelj}

\subsection{Tauri}
\subsection{Axum}
\subsection{Socketioxie}

\section{Baza podataka}

\section{Klijent}

\subsection{Pixi.js}
\subsection{React.js}
\subsection{Jotai}
\subsection{Socket.io}
\subsection{Mantine}

Ovo je glavni dio rada u kojem treba razraditi temu, pojasniti istraživanja, prikazati rezultate i slično. Poželjno je na početku poglavlja dati kratki opis strukture poglavlja, kako bi čitatelj/čitateljica rada mogao/mogla lakše pratiti složenu cjelinu.

\section{Poglavlje druge razine}

\subsection{Poglavlje treće razine}

\subsubsection{Poglavlje četvrte razine}

\chapter{Prikaz slučajeva korištenja}

Tehničke upute u nastavku opisuju način tehničkog oblikovanja rada i navođenja literature.

\section{Upute za oblikovanje izgleda rada}

\begin{flushleft}\textbf{Stranice} se oblikuju korištenjem sljedećih parametara:\end{flushleft}

\begin{itemize}
    \item veličina i oblik papira je A4, okomito usmjerenje, margine 2,5 cm na svakoj strani;

    \item naslovna stranica rada se ne numerira;

    \item nakon naslovne stranice, sve sljedeće stranice do 1. Poglavlja se numeriraju rimskim brojevima, počevši od i;

    \item od 1. poglavlja nadalje, stranice se numeriraju arapskim brojevima;

    \item broj stranice treba pozicionirati desno 1,25 cm od dna stranice, font Arial 9.
\end{itemize}
\begin{flushleft}\textbf{Tekst} rada je potrebno oblikovati sukladno ovom predlošku, odnosno na sljedeći način:\end{flushleft}
\begin{itemize}
    \item u pisanju teksta koristite font Arial 11 pt, s proredom 1,5 te razmakom 0 pt prije i razmakom 6 pt poslije odlomka, pri čemu je prvi redak uvučen za 1,25 cm;

    \item u naslovima prve razine „3. Razrada teme“ koristite font Arial 18 pt, podebljano, prijelom stranice (svaki naslov prve razine treba biti na novoj stranici), s proredom 1,5 te razmakom 0 pt prije i razmakom 18 pt poslije odlomka;

    \item u naslovima druge razine „2.1. Naslov“ koristite font Arial 16 pt, podebljano, s proredom 1,5 te razmakom 18 pt prije i razmakom 12 pt poslije odlomka;

    \item u naslovima treće razine „2.1.1. Naslov“ koristite font Arial 14 pt, podebljano, s proredom 1,5 te razmakom 12 pt prije i razmakom 6 pt poslije odlomka;

    \item u naslovima četvrte razine „2.1.1.1. Naslov“ koristite font Arial 12 pt, podebljano, s proredom 1,5 te razmakom 6 pt prije i razmakom 6 pt poslije odlomka;

    \item ostalo značajno isticanje cjelina rada može biti istaknuto podebljanim i kurziv slovima, korištenjem fonta Arial 11 pt.
\end{itemize}


\begin{flushleft}\textbf{Slike} u radu je potrebno oblikovati na sljedeći način:
naziv slike navedite ispod slike uz numeraciju;\end{flushleft}

\begin{itemize}
    \item za nazive slika koristite iste postavke fonta kao i za tekst, ali stavite naziv slike u centrirani položaj;

    \item za oblikovanje same slike koristite font Arial 9 pt za tekst na slici;
ispred same slike umetnite jedan prazan redak (osim ako je slika pozicionirana na početku stranice);

    \item nakon naziva slike ostavite jedan redak prazan (osim ako je naziv slike zadnji redak na stranici);

    \item kod prijeloma stranice treba obratiti posebnu pozornost da naziv slike, izvor i sama slika moraju biti na istoj stranici; 

    \item slike je potrebno numerirati redom pojavljivanja u tekstu;

    \item ako je slika preuzeta iz drugog izvora, nakon navođenja naziva slike u zagradi navedite izvor, npr. (autor/autorica, godina);

    \item dozvoljeno je napraviti vlastitu preradu slika, grafikona ili tablica na način da se zadrži isti smisao sadržaja, ali promijeni izgled. I u takvim se slučajevima obavezno u nazivu navodi referenca izvornog djela ovako: “(Prema: Klačmer Čalopa i Cingula, 2012)“;

    \item dozvoljeno je preuzeti samo jednu sliku, grafikon ili tablicu u izvornom obliku iz istog izvora. Za doslovno preuzimanje većeg dijela sadržaja potrebno je ishoditi dozvolu nositelja autorskih prava;

    \item primjer označavanja slike možete vidjeti u nastavku (slika \ref{fig:podjela}).
\end{itemize}

\begin{figure}[h!]
    \centering
    \includegraphics[width=0.9\textwidth]{slike/slika.png}
    \caption{Podjela investicijskih fondova (Izvor: \citeauthor{Aranda2009}, \citeyear{Aranda2009})}
    \label{fig:podjela}
\end{figure}

\begin{flushleft}\textbf{Tablice} rada je potrebno oblikovati sukladno ovim uputama:\end{flushleft}
\begin{itemize}
    \item naziv tablice navedite iznad slike;

    \item za nazive tablica koristite iste postavke fonta kao i za tekst, ali stavite naziv tablice u centrirani položaj;

    \item za oblikovanje same tablice koristite font Arial 9 pt za tekst u tablici;

    \item tablice numerirajte redom pojavljivanja u tekstu;

    \item prije naziva tablice umetnite jedan redak prazan (osim ako je naziv tablice prvi redak na stranici);

    \item nakon same tablice umetnite jedan prazan redak (osim ako je tablica pozicionirana na kraju stranice);

    \item kod prijeloma stranice treba obratiti posebnu pozornost da naziv tablice, izvor i sama tablica moraju biti na istoj stranici; 

    \item ako je tablica preuzeta iz drugog izvora, nakon navođenja naziva tablice potrebno je navesti izvor, na isti način kako je opisano kod slika;

    \item primjer označavanja tablice možete vidjeti u nastavku (tablica \ref{tab:objekti}).
\end{itemize}

\begin{table}[h!] 
    \centering
    \caption{Prikaz podataka o učestalosti pojavljivanja objekta}
    \begin{tabularx}{0.66\textwidth}{|X|X|X|X|}
        \hline
         \cellcolor{gray!25} & \cellcolor{gray!25} & \cellcolor{gray!25} & \cellcolor{gray!25} \\
        \hline
         &  &  &  \\
        \hline
         &  &  & \\
        \hline
    \end{tabularx}
    \\[10pt]
    \caption*{(Izvor: \citeauthor{caragliu2011smart}, \citeyear{caragliu2011smart})}
    \label{tab:objekti}
\end{table}

\begin{flushleft}\textbf{Programski kod}\end{flushleft}
\begin{itemize}
    \item za oblikovanje teksta koji je programski kôd koristite font Courier, veličine 10 pt, jednostruki prored, 6 pt iza odlomka, npr. HTML kôd dijela zaglavlja početne web stranice FOI weba:
\end{itemize}

\begin{lstlisting}[language=HTML]
<head>
  <meta http-equiv="Content-Type" content="text/html; charset=utf-8" />
  <link rel="shortcut icon" href="https://www.foi.unizg.hr/sites/default/files/favicon_0_1.ico" type="image/vnd.microsoft.icon" />
  <meta name="generator" content="Drupal 7 (http://drupal.org)" />
  <link rel="canonical" href="https://www.foi.unizg.hr/hr" />
  <link rel="shortlink" href="https://www.foi.unizg.hr/hr" />
  <!-- Set the viewport width to device width for mobile -->
  <meta name="viewport" content="width=device-width, initial-scale=1.0">
  <title>Dobro %*došli*) na FOI | FOI</title>...
</head>
\end{lstlisting}

\begin{flushleft}\textbf{Formule}\end{flushleft}
\begin{itemize}
    \item za unos formula koristite editor za formule u svom tekst procesoru.
\end{itemize}

\begin{flushleft}\textbf{Kratice}\end{flushleft}   
\begin{itemize}
    \item ako želite koristiti kratice pojmova u tekstu, kad prvi put spominjete pojam potrebno je navesti puni naziv, a kraticu navesti u zagradi (npr. Informacijske i komunikacijske tehnologije, kraće IKT). Nakon toga možete koristiti kratice u tekstu. Poželjno je u naslovima koristiti pune nazive.
\end{itemize}

\begin{flushleft}\textbf{Strano nazivlje}\end{flushleft}   
\begin{itemize}
    \item strano nazivlje se u tekstu navodi u zagradi, napisano \textit{kurzivom}, nakon hrvatskog izraza, npr. Analiza društvene mreže (engl. \textit{Social Network Analysis - SNA}).
\end{itemize}

\section{Navođenje literature}

Za navođenje literature u radu možete odabrati i koristiti jedan od sljedeća dva ponuđena stila: \textbf{APA} ili \textbf{IEEE} stil. Važno je samo dosljedno primjenjivati odabrani stil u cijelom radu.

U popisu literature potrebno je navesti svu literaturu i samo literaturu koju ste koristili u tekstu.

Uz svaku preuzetu tvrdnju potrebno je navesti njezin izvor, tj. referencu. Reference se u tekstu navode tako da se uz citirani tekst navede izvor, sukladno načinu propisanom odabranim stilom i FOI preporukama za citiranje i referenciranje \cite{SchattenEtAl2016roadmap}.

\chapter{Zaključak}

Ovdje treba sažeto rezimirati najvažnije rezultate razrade teme rada. Potrebno je sažeto opisati što je predmet rada, koje su metode, tehnike, programski alati ili aplikacije korištene u razradi rada te koje su pretpostavke dokazane, a koje opovrgnute. Sadržajno, ono što se u uvodu rada najavljuje i kasnije je obuhvaćeno u samom radu, moralo bi biti opisano u zaključnom dijelu kroz rezultate rada. 

\lipsum[1-2]

\printbibliography[title=Popis literature]
\addcontentsline{toc}{chapter}{Popis literature}

\listoffigures
\addcontentsline{toc}{chapter}{Popis slika}
 
\listoftables
\addcontentsline{toc}{chapter}{Popis popis tablica}

\appendix
\renewcommand{\thechapter}{\arabic{chapter}}

\end{document}
